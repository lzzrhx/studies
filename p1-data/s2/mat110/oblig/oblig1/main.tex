\documentclass[norsk,8pt,a4paper]{report}
\usepackage[margin=1.5cm,tmargin=1.0cm,bmargin=1.0cm,rmargin=1.5cm,lmargin=1.5cm,footskip=0.2cm]{geometry}
\title{MAT110 - Obligatorisk innlevering 1}
\author{Gruppemedlemmer:\\Stephen Neba Fuh, Tord Johan Melheim,\\Ebubekir Siddik Yuksel, Casper Eide Özdemir-Børretzen }
\date{}

% % % % % % % % % % % % % % % % % % % % % % % % % % % % % % % % % % % % % % % % 

\usepackage{nicematrix}
\usepackage{tikz}
\usepackage{babel}               % Support for other languages than english
\usepackage{setspace}            % Set paragraph spacing
\usepackage{nicefrac}            % Nice looking fractals
\usepackage{graphicx}            % Images
\usepackage{gensymb}             % Degree symbol
\usepackage{listings}            % Code
\usepackage{ulem}                % Double underline
\usepackage{amssymb}             % ...
%\usepackage{pdfpages}            % Insert pdf pages
\usepackage{enumitem}            % Lists
\usepackage{colortbl}            % Colored tables
\usepackage[compact]{titlesec}   % ...
\usepackage[T1]{fontenc}         % ...
\usepackage[utf8]{inputenc}      % ...
\usepackage[fleqn]{amsmath}      % ...
\usepackage[makeroom]{cancel}    % ...
%\usepackage{empheq}              % ...
%\setstretch{1.5}
\setlength{\parindent}{0pt}
\titlespacing*{\subsection}{0cm}{1.5cm}{0.5cm}
\lstset{
aboveskip=0.1cm,
belowskip=1.0cm,
showstringspaces=false,
columns=flexible,
basicstyle={\scriptsize\ttfamily},
breaklines=true,
breakatwhitespace=true,
tabsize=4,
inputencoding = utf8,  % Input encoding
extendedchars = true,  % Extended ASCII
literate      =        % Support additional characters
{⋆}{{$\star\ $}}1 {∘}{{$\circ\ $}}1
{á}{{\'a}}1  {é}{{\'e}}1  {í}{{\'i}}1 {ó}{{\'o}}1  {ú}{{\'u}}1
{Á}{{\'A}}1  {É}{{\'E}}1  {Í}{{\'I}}1 {Ó}{{\'O}}1  {Ú}{{\'U}}1
{à}{{\`a}}1  {è}{{\`e}}1  {ì}{{\`i}}1 {ò}{{\`o}}1  {ù}{{\`u}}1
{À}{{\`A}}1  {È}{{\`E}}1  {Ì}{{\`I}}1 {Ò}{{\`O}}1  {Ù}{{\`U}}1
{ä}{{\"a}}1  {ë}{{\"e}}1  {ï}{{\"i}}1 {ö}{{\"o}}1  {ü}{{\"u}}1
{Ä}{{\"A}}1  {Ë}{{\"E}}1  {Ï}{{\"I}}1 {Ö}{{\"O}}1  {Ü}{{\"U}}1
{â}{{\^a}}1  {ê}{{\^e}}1  {î}{{\^i}}1 {ô}{{\^o}}1  {û}{{\^u}}1
{Â}{{\^A}}1  {Ê}{{\^E}}1  {Î}{{\^I}}1 {Ô}{{\^O}}1  {Û}{{\^U}}1
{œ}{{\oe}}1  {Œ}{{\OE}}1  {æ}{{\ae}}1 {Æ}{{\AE}}1  {ß}{{\ss}}1
{ẞ}{{\SS}}1  {ç}{{\c{c}}}1 {Ç}{{\c{C}}}1 {ø}{{\o}}1  {Ø}{{\O}}1
{å}{{\aa\ }}1  {Å}{{\AA}}1  {ã}{{\~a}}1  {õ}{{\~o}}1 {Ã}{{\~A}}1
{Õ}{{\~O}}1  {ñ}{{\~n}}1  {Ñ}{{\~N}}1  {¿}{{?`}}1  {¡}{{!`}}1
{„}{\quotedblbase}1 {“}{\textquotedblleft}1 {–}{$-$}1
{°}{{\textdegree}}1 {º}{{\textordmasculine}}1 {ª}{{\textordfeminine}}1
{£}{{\pounds}}1  {©}{{\copyright}}1  {®}{{\textregistered}}1
{«}{{\guillemotleft}}1  {»}{{\guillemotright}}1  {Ð}{{\DH}}1  {ð}{{\dh}}1
{Ý}{{\'Y}}1    {ý}{{\'y}}1    {Þ}{{\TH}}1    {þ}{{\th}}1    {Ă}{{\u{A}}}1
{ă}{{\u{a}}}1  {Ą}{{\k{A}}}1  {ą}{{\k{a}}}1  {Ć}{{\'C}}1    {ć}{{\'c}}1
{Č}{{\v{C}}}1  {č}{{\v{c}}}1  {Ď}{{\v{D}}}1  {ď}{{\v{d}}}1  {Đ}{{\DJ}}1
{đ}{{\dj}}1    {Ė}{{\.{E}}}1  {ė}{{\.{e}}}1  {Ę}{{\k{E}}}1  {ę}{{\k{e}}}1
{Ě}{{\v{E}}}1  {ě}{{\v{e}}}1  {Ğ}{{\u{G}}}1  {ğ}{{\u{g}}}1  {Ĩ}{{\~I}}1
{ĩ}{{\~\i}}1   {Į}{{\k{I}}}1  {į}{{\k{i}}}1  {İ}{{\.{I}}}1  {ı}{{\i}}1
{Ĺ}{{\'L}}1    {ĺ}{{\'l}}1    {Ľ}{{\v{L}}}1  {ľ}{{\v{l}}}1  {Ł}{{\L{}}}1
{ł}{{\l{}}}1   {Ń}{{\'N}}1    {ń}{{\'n}}1    {Ň}{{\v{N}}}1  {ň}{{\v{n}}}1
{Ő}{{\H{O}}}1  {ő}{{\H{o}}}1  {Ŕ}{{\'{R}}}1  {ŕ}{{\'{r}}}1  {Ř}{{\v{R}}}1
{ř}{{\v{r}}}1  {Ś}{{\'S}}1    {ś}{{\'s}}1    {Ş}{{\c{S}}}1  {ş}{{\c{s}}}1
{Š}{{\v{S}}}1  {š}{{\v{s}}}1  {Ť}{{\v{T}}}1  {ť}{{\v{t}}}1  {Ũ}{{\~U}}1
{ũ}{{\~u}}1    {Ū}{{\={U}}}1  {ū}{{\={u}}}1  {Ů}{{\r{U}}}1  {ů}{{\r{u}}}1
{Ű}{{\H{U}}}1  {ű}{{\H{u}}}1  {Ų}{{\k{U}}}1  {ų}{{\k{u}}}1  {Ź}{{\'Z}}1
{ź}{{\'z}}1    {Ż}{{\.Z}}1    {ż}{{\.z}}1    {Ž}{{\v{Z}}}1  {ž}{{\v{z}}}1
}

\newlength{\myeqskip}  \setlength{\myeqskip}{2pt}

% % % % % % % % % % % % % % % % % % % % % % % % % % % % % % % % % % % % % % % % 

\newcommand{\idots}{\mathrel{{.}\,{.}}\nobreak}
\newcommand{\oppgave}[1]{\subsection*{Oppgave #1}}
\newcommand{\oppgaveDelStart}{\begin{enumerate}[leftmargin=*,itemsep=1.5cm,labelsep=1.5em,label=\alph*)]}
\newcommand{\oppgaveDelSlutt}{\end{enumerate}}
\newcommand{\oppgaveDel}[1]{\item[#1)]}

% % % % % % % % % % % % % % % % % % % % % % % % % % % % % % % % % % % % % % % % 

\begin{document}

% % % % % % % % % % % % % % % % % % % % % % % % % % % % % % % % % % % % % % % % 

\maketitle

% % % % % % % % % % % % % % % % % % % % % % % % % % % % % % % % % % % % % % % % 

\oppgave{1}
\oppgaveDelStart
\oppgaveDel{a} \ \\

\usetikzlibrary {angles,quotes}


\begin{center}
\begin{tikzpicture}[scale=2.00]
%\draw[step=1cm,gray,very thin] (-5,-5) grid (5,5);
\draw(2.5,0) node{Re};
\draw(0,2.5) node{Im};
%\draw[--] (0,0) -- (2,1) -- (3,3) -- (4,2) -- (5,1.5) -- (7,2) -- (8,2.5);
\draw[black!50](-1,0) coordinate (A) -- (0,0) coordinate (B) -- (-1,-1) coordinate (C) pic [black!50, "$\alpha$", draw, angle eccentricity=1.5] {angle};
\foreach \x/\xtext in { -2, -1, 0.7071, 2}
    \draw [ shift={(\x,0)}]
    (0pt,0pt) -- (0pt, -2pt)
    node [below] {\tiny$\xtext$};
\foreach \y/\ytext in {0.7071, 2}
    \draw [ shift={(0,\y)}]
    (0pt,0pt) -- (-2pt, 0pt)
    node [left] {\tiny$\ytext$};
\foreach \y/\ytext in {-2, -1}
    \draw [ shift={(0,\y)}]
    (0pt,0pt) -- (2pt, 0pt)
    node [right] {\tiny$ \ \ytext$};
\draw[black!30, dotted] (0.7071,0) -- (0.7071,0.7071);
\draw[black!30, dotted] (0,0.7071) -- (0.7071,0.7071);
\draw[black!30, dotted] (0,-1) -- (-1,-1);
\draw[black!30, dotted] (-1,0) -- (-1,-1);
\draw[thick,->](-2.1,0) -- (2.1,0);
\draw[thick,->](0,-2.1) -- (0,2.1);
\node (z2) at (-1,-1) {$\bullet$};
\node at (0.7071,0.7071) {$\bullet$};
\draw(-1,-1+0.25) node{$z_2$};
\draw(0.7071,0.7071+0.25) node{$z_1$};

\end{tikzpicture}
\end{center}

\oppgaveDel{b}

\begin{alignat*}\\
z_1 = & \ e^{i\frac{\pi}{4}} \ \ \text{(eksponentialform)} \rightarrow r=1, \ \ \theta = \frac{\pi}{4}\\
= & \ 1(\cos \theta + i \cdot \sin \theta)\\
= & \ cos \frac{\pi}{4} + i \cdot sin \frac{\pi}{4}\\
= & \ x_1 + i \cdot y_1\\
\Rightarrow x_1 = &\cos \frac{\pi}{4} = \frac{1}{\sqrt{2}} = \frac{\sqrt{2}}{2}\\
\Rightarrow y_1 = &\sin \frac{\pi}{4} = \frac{1}{\sqrt{2}} = \frac{\sqrt{2}}{2}\\
z_1 = &\frac{\sqrt{2}}{2} + i \cdot \frac{\sqrt{2}}{2} \ \ \text{(standardform)}\\\\
z_{2} &= -1 - i \ \ \text{(standardform)}\\
&= x_2 + i \cdot y_2 \rightarrow x_2 = -1, \ \ y_2 = -1 \\
&r = \sqrt{x_2^2 + y_2^2} = \sqrt{(-1)^2 + (-1)^2} = \sqrt{2}\\
&\theta = \tan^{-1} (| \frac{y}{x} |) \ \ \text{(Generelt)}\\
&\text{Men ettersom det komplekse tallet er i 3. kvadrant} \rightarrow \theta = \alpha + \pi\\
&\alpha = \tan^{-1} (| \frac{-1}{-1} |) = \tan^{-1} (1) = \frac{\pi}{4}\\
&\theta = \pi + \frac{\pi}{4} = \frac{5 \pi}{4}\\
z_2 &= r (\cos \theta + i \cdot \sin \theta)\\
z_2 &= \sqrt{2} (\cos \frac{5 \pi}{4} + i \cdot \frac{5 \pi}{4})\\
z_2 &= \sqrt{2} \cdot e^{i \frac{5 \pi}{4}} \ \ \text{(eksponentialform)}
\end{alignat*}

\oppgaveDelSlutt
\newpage
\oppgave{1}
\oppgaveDelStart

\oppgaveDel{c}
\begin{alignat*}\\
z_1 + z_2 &= (\frac{\sqrt{2}}{2} + \sqrt{2}{2} \cdot i) + (-1 - i)\\
          &= (\frac{\sqrt{2}}{2} - 1) + (\frac{\sqrt{2}}{2} - 1) \cdot i\\
&=(0.7071 - 1) + (0.7071 - 1) \cdot i\\
&=-0.29 - 0.29 i\\ \\ \\
\frac{z_1}{z_2} &= \frac{e^{i \frac{\pi}{4}}}{\sqrt{2} e^{i \frac{5 \pi}{4}}} = \frac{1}{\sqrt{2}} e^{i \frac{\pi}{4} - i \frac{t \pi}{4}} = \frac{1}{\sqrt{2}} e^{-i \pi}\\
&= \frac{1}{\sqrt{2}}( \cos (- \pi) + i \sin (- \pi)) = \frac{1}{\sqrt{2}} (-1 + i (0)) = - \frac{1}{\sqrt{2}} + i0\\
&= - \frac{\sqrt{2}}{2}\\ \\ \\
{z_1} &= \frac{\sqrt{2}}{2} + i \frac{\sqrt{2}}{2}\\
\overline{z_1} &= \frac{\sqrt{2}}{2} - i \frac{\sqrt{2}}{2}\\
- \overline{z_1} &= -1 ( \frac{\sqrt{2}}{2} - i \frac{\sqrt{2}}{2} ) = - \frac{\sqrt{2}}{2} + i \frac{\sqrt{2}}{2} \\
\end{alignat*}

\oppgaveDel{d}
$z^6 = 1 + i$. La $w = 1 + i$, da må vi finne den 6. roten av $w$.\\
Først konverterer vi $w$ til eksponentialform.\\
$r = \sqrt{1^2 + 1^2} = \sqrt{2}$\\
$\theta = \tan^{-1} (1) = \frac{\pi}{4}$\\
$w = 1 + i = \sqrt{2} e^{i \frac{\pi}{4}}$\\\\
Videre tar vi i bruk De Moivres thm. for røtter:\\
Hvis $z^n = w = r e^{i \theta}$, da er røttene gitt ved $z_k = \sqrt[n]{r} e^{i (\frac{\theta + 2 \pi k}{n})}$, for $k = 0, 1, 2, \dots, n-1$.\\\\
I dette tilfellet blir $n=6, r=\sqrt{2}$ og $\theta = \frac{\pi}{4}$.\\
$\sqrt[6]{r} = \sqrt[6]{\sqrt{2}} = \sqrt[6]{2^{\frac{1}{2}}} = 2^{(\frac{1}{2})(\frac{1}{6})} = 2^{\frac{1}{12}} = \sqrt[12]{2}$.\\
Videre finner vi argumentene for $k = 0,1,2,3,4,5$:\\
Det generelle argumentet er $\phi_k = \frac{\frac{\pi}{4} + 2 \pi k}{6} = \frac{\pi}{24} + \frac{2 \pi k}{6} = \frac{\pi}{24} + \frac{\pi k}{3}.$

\begin{alignat*}\\
&k = 0: \phi_0 = \frac{\pi}{24}\\
&k = 1: \phi_1 = \frac{\pi}{24} + \frac{\pi}{3} = \frac{9 \pi}{24}\\
&k = 2: \phi_2 = \frac{\pi}{24} + \frac{2 \pi}{3} = \frac{17 \pi}{24}\\
&k = 3: \phi_3 = \frac{\pi}{24} + \pi = \frac{25 \pi}{24}\\
&k = 4: \phi_4 = \frac{\pi}{24} + \frac{4 \pi}{3} = \frac{33 \pi}{24}\\
&k = 5: \phi_5 = \frac{\pi}{24} + \frac{5 \pi}{3} = \frac{41 \pi}{24}\\\\
&\text{Det er seks løsninger:}\\
&z_k = 2^{\frac{1}{12}} \cdot e^{i \phi_k}\text{, for }k = 0,1, 2, 3, 4, 5.
\end{alignat*}


\oppgaveDelSlutt

% % % % % % % % % % % % % % % % % % % % % % % % % % % % % % % % % % % % % % % % 

\oppgave{2}
\oppgaveDelStart

\oppgaveDel{a}

$\begin{bmatrix}
1  & 0 &     2k & k & 2\\
1  & 1 & (2k-1) & 0 & 1\\
-2 & 0 &    -4k & k & 5\\
\end{bmatrix}
\sim\begin{matrix}
\\
R_2 - R_1\\
\\ 
\end{matrix}
\begin{bmatrix}
1  & 0 &  2k &  k &  2\\
0  & 1 &  -1 & -k & -1\\
-2 & 0 & -4k &  k &  5\\
\end{bmatrix}
\sim\begin{matrix}
\\
\\
R_3 + 2 R_1\\ 
\end{matrix}
\begin{bmatrix}
1 & 0 & 2k &  k &  2\\
0 & 1 & -1 & -k & -1\\
0 & 0 &  0 & 3k &  9\\
\end{bmatrix}$

\begin{alignat*}\\
& (1) \ \ x_1 + 2k x_3 + k x_4 &&= 2\\
& (2) \ \ x_2 - x_3 - k x_4 &&= -1\\
& (3) \ \ 3k x_4 &&= 9
\end{alignat*}

\oppgaveDel{b}
Fra $(3) \ \ 3k x_4 = 9$\\\\
Case 1:\\
$k \neq 0 \Rightarrow x_4 = \frac{3}{k}$\\\\
Vi setter $x_4 = \frac{4}{k}$ inn i $(2)$:\\
$x_2 = x_3 + k \frac{3}{k} - 1 = x_3 + 2$\\
Så vi trenger verdien til $x_3$ for å beregne $x_2$.\\\\
Vi setter $x_4 = \frac{3}{k}$ inn i $(1)$:\\
$x_1 + 2k x_3 + k \frac{3}{k} = x_1 + 2k x_3 + 3 = 2$\\
$\Rightarrow x_1 = -1 - 2k x_3$\\\\
For $x \neq 0$, $x_1$ og $x_2$ er uttrykt med bruk av den frie variablen $x_3$.\\
Derfor er det uendelig mange løsninger for alle $k \neq 0$.\\\\

Case 2:\\
$k = 0 \Rightarrow (3)$ blir $0 \cdot x_4 = 9 \Rightarrow 0 = 9$.\\

Dette er umulig og vi kan konkludere at det er ingen løsning for $k = 0$.\\

Utifra dette kan vi slå fast at det kan ikke være akkurat en eller akkurat to løsninger i dette tilfelle, enten uendelig mange løsninger (for $k \neq 0$) eller ingen løsning (for $k = 0$).
\oppgaveDel{c}

\begin{alignat*}\\
&(3) \ \ (3)(6) x_4 = 9 \Rightarrow 18 x_4 = 9 \Rightarrow x_4 = \frac{1}{2}\\
&(2) \ \ x_2 - x_3 - 6(\frac{1}{2}) = -1 \Rightarrow x_3 - x_3 - 3 = -1 \Rightarrow x_2 = x_3 + 2\\
&(1) \ \ x_1 + 2(6) x_3 + 6(\frac{1}{2}) = 2 \Rightarrow x_1 + 12 x_3 + 3 = 2 \Rightarrow x_1 = -1 - 12 x_3\\
&\therefore \ \ (x_1, x_2, x_3,x_4) = (-1 -12t, t+2, t,\frac{1}{2}), t \in \mathbb{R} \ \ \text{(Uendelig mange løsninger)}
\end{alignat*}\\

\oppgaveDelSlutt

% % % % % % % % % % % % % % % % % % % % % % % % % % % % % % % % % % % % % % % % 

\newpage
\oppgave{3}
\oppgaveDelStart

$A =
\begin{bmatrix}
1 & 1 &  1\\
1 & 2 & -1\\
2 & 3 &  3\\
\end{bmatrix},
 \ \ B =
\begin{bmatrix}
 1 &  1\\
 1 & -1\\
-1 &  2\\
\end{bmatrix}$

\oppgaveDel{a}

$A + B$ er ikke mulig ettersom matrisene har ulike dimensjoner.\\\\

$3A = 
\begin{bmatrix}
3 & 3 &  3\\
3 & 6 & -3\\
6 & 9 &  9\\
\end{bmatrix},
 \ \ B^T = 
\begin{bmatrix}
1 & -1 &  2\\
1 &  1 & -1\\
\end{bmatrix}$\\

$3A - B^T$ er ikke mulig ettersom matrisene har ulike dimensjoner.\\\\

$AB =
\begin{bmatrix}
1 & 1 &  1\\
1 & 2 & -1\\
2 & 3 &  3\\
\end{bmatrix}
\cdot
\begin{bmatrix}
 1 &  1\\
 1 & -1\\
-1 &  2\\
\end{bmatrix} = 
\begin{bmatrix}
1*1+1*1+1(-1) & 1*1+1(-1)+1*2\\
1*1+2*1-1(-1) & 1*1+2(-1)-1*2\\
2*1+3*1+3(-1) & 2*1+3(-1)+3*2\\
\end{bmatrix} = 
\begin{bmatrix}
1 & 2\\
4 & -3\\
2 & 5\\
\end{bmatrix}
$\\\\

$BA$  er ikke mulig ettersom dimensjonene på matrisene må være $m \times p \cdot p \times n$ for å gjennomføre multiplikasjon.\\\\

B^T A = 
\begin{bmatrix}
1 &  1 & -1\\
1 & -1 &  2\\
\end{bmatrix}$
\cdot
\begin{bmatrix}
1 & 1 &  1\\
1 & 2 & -1\\
2 & 3 &  3\\
\end{bmatrix} = 
\begin{bmatrix}
1(1)+1(1)-1(2) & 1(1)+1(2)-1(3) & 1(1)+1(-1)-1(3)\\
1(1)-1(1)+2(2) & 1(1)-1(2)+2(3) & 1(1)-1(-1)+2(3)\\
\end{bmatrix} = 
\begin{bmatrix}
0 & 0 & -3\\
4 & 5 & 8\\
\end{bmatrix}

\oppgaveDel{b}

$\det(A) = 
1 \cdot
\begin{vmatrix}
2 & -1\\
3 &  3\\
\end{vmatrix}
- 1 \cdot
\begin{vmatrix}
1 & -1\\
2 &  3\\
\end{vmatrix}
+ 1 \cdot
\begin{vmatrix}
1 & 2\\
2 & 3\\
\end{vmatrix} = 1 \cdot (6+3) -1(3+2) + 1(3-4) = 9-5-1 = 3$

\oppgaveDel{c}

\renewcommand{\arraystretch}{1.4}
$\sim\begin{NiceArray}{W{c}{2cm}c}
\\
\\
\\ 
\end{NiceArray}
\begin{bNiceArray}{*{3}{c}|*{3}{c}}[columns-width=0.5cm]
1 & 1 &  1 & 1 & 0 & 0\\
1 & 2 & -1 & 0 & 1 & 0\\
2 & 3 &  3 & 0 & 0 & 1\\
\end{bNiceArray}
\sim\begin{NiceArray}{W{c}{2cm}c}
\\
R_2 - R_1\\ 
\\
\end{NiceArray}
\begin{bNiceArray}{*{3}{c}|*{3}{c}}[columns-width=0.5cm]
1 & 1 &  1 &  1 & 0 & 0\\
0 & 1 & -2 & -1 & 1 & 0\\
2 & 3 &  3 &  0 & 0 & 1\\
\end{bNiceArray}$\\\\

$\sim\begin{NiceArray}{W{c}{2cm}c}
\\
\\
R_3 - 2 R_1\\ 
\end{NiceArray}
\begin{bNiceArray}{*{3}{c}|*{3}{c}}[columns-width=0.5cm]
1 & 1 &  1 & 1 & 0 & 0\\
0 & 1 & -2 &-1 & 1 & 0\\
0 & 1 &  1 &-2 & 0 & 1\\
\end{bNiceArray}
\sim\begin{NiceArray}{W{c}{2cm}c}
\\
\\
R_3 - R_2\\ 
\end{NiceArray}
\begin{bNiceArray}{*{3}{c}|*{3}{c}}[columns-width=0.5cm]
1 & 1 &  1 & 1 &  0 & 0\\
0 & 1 & -2 &-1 &  1 & 0\\
0 & 0 &  3 &-1 & -1 & 1\\
\end{bNiceArray}$\\\\

$\sim\begin{NiceArray}{W{c}{2cm}c}
\\
\\
\frac{1}{3} R_3\\ 
\end{NiceArray}
\begin{bNiceArray}{*{3}{c}|*{3}{c}}[columns-width=0.5cm]
1 & 1 &  1 & 1 &  0 & 0\\
0 & 1 & -2 &-1 &  1 & 0\\
0 & 0 &  1 &- \frac{1}{3} & - \frac{1}{3} & \frac{1}{3}\\
\end{bNiceArray}
\sim\begin{NiceArray}{W{c}{2cm}c}
\\
R_2 + 2 R_3\\
\\
\end{NiceArray}
\begin{bNiceArray}{*{3}{c}|*{3}{c}}[columns-width=auto]
1 & 1 & 1 & 1 &  0 & 0\\
0 & 1 & 0 &- \frac{5}{3} & \frac{1}{3} & \frac{2}{3}\\
0 & 0 & 1 &- \frac{1}{3} & - \frac{1}{3} & \frac{1}{3}\\
\end{bNiceArray}$\\\\

$\sim\begin{NiceArray}{W{c}{2cm}c}
R_1 - R_2\\
\\
\\
\end{NiceArray}
\begin{bNiceArray}{*{3}{c}|*{3}{c}}[columns-width=0.5cm]
1 & 0 & 1 & \frac{8}{3} & - \frac{1}{3} & - \frac{2}{3}\\
0 & 1 & 0 &- \frac{5}{3} & \frac{1}{3} & \frac{2}{3}\\
0 & 0 & 1 &- \frac{1}{3} & - \frac{1}{3} & \frac{1}{3}\\
\end{bNiceArray}
\sim\begin{NiceArray}{W{c}{2cm}c}
R_1 - R_3\\
\\
\\
\end{NiceArray}
\begin{bNiceArray}{*{3}{c}|*{3}{c}}[columns-width=0.5cm]
1 & 0 & 0 & 3 & 0 & -1\\
0 & 1 & 0 &- \frac{5}{3} & \frac{1}{3} & \frac{2}{3}\\
0 & 0 & 1 &- \frac{1}{3} & - \frac{1}{3} & \frac{1}{3}\\
\end{bNiceArray}$\\\\

\begin{center}
A^{-1} = 
\begin{bNiceArray}{*{3}{c}}[columns-width=0.5cm]
3 & 0 & -1\\
-\frac{5}{3} & \frac{1}{3} & \frac{2}{3}\\
-\frac{1}{3} & - \frac{1}{3} & \frac{1}{3}\\
\end{bNiceArray}
\end{center}


\oppgaveDelSlutt
\newpage
\oppgave{3}
\oppgaveDelStart
\oppgaveDel{d}

\begin{equation*}
\end{equation*}

\begin{alignat*}\\
&A \vec{x} = \vec{b}\\
&\vec{b} = \begin{bmatrix}1\\4\\2\end{bmatrix}\\
&\Rightarrow \vec{x} = A^{-1} \vec{b} =
\begin{bmatrix}
3 & 0 & -1\\
-\frac{5}{3} & \frac{1}{3} & \frac{2}{3}\\
-\frac{1}{3} & - \frac{1}{3} & \frac{1}{3}\\
\end{bmatrix}\cdot\begin{bmatrix}1\\4\\2\end{bmatrix}\\\\
& 3(1) + 0(4) - 1(2) = 3-2 &&= 1\\\\
& - \frac{5}{3}(1) + \frac{1}{3}(4) + \frac{2}{3}(2) = - \frac{5}{3} + \frac{4}{3} + \frac{4}{3} &&= 1\\\\
& - \frac{1}{3}(1) - \frac{1}{3}(4) + \frac{1}{3}(2) = - \frac{1}{3} - \frac{4}{3} + \frac{2}{3} &&= -1\\\\
&\Rightarrow x_1 = 1, \ \ x_2 = 1, \ \ x_3 = -1
\end{alignat*}

\oppgaveDel{e}

\lstinputlisting[language=Python]{3e.py}

\oppgaveDelSlutt

% % % % % % % % % % % % % % % % % % % % % % % % % % % % % % % % % % % % % % % % 

\oppgave{4}
\begin{alignat*}
&\vec{u} = \left[ \frac{1}{\sqrt{2}}, \ \frac{1}{2}, \ \frac{1}{2} \right], \ \ \vec{v} = \left[ - \frac{1}{\sqrt{2}}, \ \frac{1}{2}, \ \frac{1}{2} \right], \ \ \vec{w} = \left[ - \frac{1}{\sqrt{2}}, \ 0, \ \frac{1}{\sqrt{2}} \right]
\end{alignat*}
\oppgaveDelStart

\oppgaveDel{a}
\begin{alignat*}\\
& |\vec{u}| = \sqrt{(\nicefrac{1}{\sqrt{2}})^2 + (\nicefrac{1}{2})^2 + (\nicefrac{1}{2})^2 } &&= \sqrt{\nicefrac{1}{2} + \nicefrac{1}{4} + \nicefrac{1}{4}} &&&= \sqrt{1} = 1\\\\
& |\vec{v}| = \sqrt{(- \nicefrac{1}{\sqrt{2}})^2 + (\nicefrac{1}{2})^2 + (\nicefrac{1}{2})^2 } &&= \sqrt{\nicefrac{1}{2} + \nicefrac{1}{4} + \nicefrac{1}{4}} &&&= \sqrt{1} = 1\\\\
& |\vec{w}| = \sqrt{(- \nicefrac{1}{\sqrt{2}})^2 + 0^2 + (\nicefrac{1}{sqrt{2}})^2 } &&= \sqrt{\nicefrac{1}{2} + \nicefrac{1}{2}} &&&= \sqrt{1} = 1
\end{alignat*}

\oppgaveDelSlutt
\newpage
\oppgave{4}
\oppgaveDelStart
\oppgaveDel{b}

\begin{alignat*}\\
&\vec{u} \cdot \vec{v} = \nicefrac{1}{\sqrt{2}} \cdot (\nicefrac{-1}{\sqrt{2}}) + \nicefrac{1}{2} \cdot \nicefrac{1}{2} + \nicefrac{1}{2} \cdot \nicefrac{1}{2} = - \nicefrac{1}{2} + \nicefrac{1}{4} + \nicefrac{1}{4} = 0\\
&\vec{u} \cdot \vec{v} = |\vec{u}| \cdot |\vec{v}| \cos \theta = 1 \cdot 1 \cdot \cos \theta = \cos \theta = 0\\
&\cos \theta = 0 \Rightarrow \theta = 90 \degree \ \ (\vec{u} \ \bot \ \vec{v})
\end{alignat*}

\begin{alignat*}\\
&\vec{u} \cdot \vec{w} = \frac{1}{\sqrt{2}} \cdot (\frac{-1}{\sqrt{2}}) + \frac{1}{2} \cdot 0 + \frac{1}{2} \cdot \frac{1}{\sqrt{2}} = - \frac{1}{2} + \frac{1}{2 \sqrt{2}}\\
&\Rightarrow \cos \theta = \frac{1}{2 \sqrt{2}} - \frac{1}{2}\\
&\Rightarrow \theta = \cos^{-1} ( \frac{1}{2 \sqrt{2}} - \frac{1}{2} ) \approx 98,4 \degree
\end{alignat*}

\oppgaveDel{c}

\begin{alignat*}\\
&\vec{u} \times \vec{v} = 
\begin{bmatrix}
i & j & k\\
\frac{1}{\sqrt{2}} & \frac{1}{2} & \frac{1}{2}\\
- \frac{1}{\sqrt{2}} & \frac{1}{2} & \frac{1}{2}\\
\end{bmatrix}\\\\
&i: \ \frac{1}{2} \cdot \frac{1}{2} - \frac{1}{2} \cdot \frac{1}{2} &&= 0\\\\
&j: \ -\left(\frac{1}{\sqrt{2}} \cdot \frac{1}{2} - \frac{1}{2} \cdot (-\frac{1}{\sqrt{2}}) \right) &&= - \frac{1}{\sqrt{2}}\\\\
&k: \ \frac{1}{\sqrt{2}} \cdot \frac{1}{2} - \frac{1}{2} \cdot (- \frac{1}{\sqrt{2}}) &&= \frac{1}{\sqrt{2}}\\\\
&\vec{u} \times \vec{v} = \left[ 0, \ \ - \frac{1}{\sqrt{2}}, \ \ \frac{1}{\sqrt{2}}\right]
\end{alignat*}

\oppgaveDel{d}
$\vec{u} \times \vec{v}$ er vinkelrett til både $\vec{a}$ og $\vec{b}$, så vinkelen mellom $\vec{a}$ og $\vec{u} \times \vec{v}$ er $90 \degree$.

\oppgaveDel{e}

\begin{alignat*}\\
&\vec{a} \left[ 5\vec{a} + \vec{b} + (\vec{a} \times \vec{b}) \right] = 5 (\vec{a} \cdot \vec{a} + \vec{a} \cdot \vec{b} + \vec{a} \cdot (\vec{a} \times \vec{b}))\\
&\text{Vi vet at:}\\
(i) & \ \vec{a} \cdot \vec{a} = |\vec{a}|^2 = 2^2 &= 4\\
(ii) & \ \vec{a} \cdot \vec{b} = 0 \ \text{(ettersom $\vec{a}$ og $\vec{b}$ står vinkelrett på hverandre)}\\
(iii) & \ \vec{a} \cdot (\vec{a} \times \vec{b}) = 0 \ \text{(ettersom kryssproduktet til to vektorer står vinkelrett på begge vektorer)}\\
& \therefore \ \vec{a} \left[ 5\vec{a} + \vec{b} + (\vec{a} \times \vec{b}) \right] = 5(4) + 0 + 0 = 20
\end{alignat*}

\oppgaveDelSlutt

% % % % % % % % % % % % % % % % % % % % % % % % % % % % % % % % % % % % % % % % 

\oppgave{5}
\oppgaveDelStart

\oppgaveDel{a}\\
\begin{alignat*}\\
&50 x_1 +  25 x_2 +  75 x_3 + 100 x_4 = 4200\\
&75 x_1 + 100 x_2 +  50 x_3 +  25 x_4 = 3800\\
&75 x_1 +  75 x_2 +  75 x_3 +  50 x_4 = 4400\\
&50 x_1 +  50 x_2 +  50 x_3 +  75 x_4 = 3600\\\\
&x_1 \text{ er antall produserte Floral Fusion blandinger.}\\
&x_2 \text{ er antall produserte Burgundy Bonanza blandinger.}\\
&x_3 \text{ er antall produserte Morgensol blandinger.}\\
&x_4 \text{ er antall produserte Ahh Svart! blandinger.}\\
\end{alignat*}

\oppgaveDel{b}
$A = 
\begin{bmatrix}
 50 &  25 &  75 & 100\\
 75 & 100 &  50 &  25\\
 75 &  75 &  75 &  50\\
 50 &  50 &  50 &  75\\
\end{bmatrix}\\\\

\lstinputlisting[language=Python]{5b.py}

Hvis $A$ har en invers $A^{-1}$ kan inversen brukes for å løse likningssystemet ($\vec{b} = A^{-1} \vec{x}$), men $\det(A) = 0$ forteller oss at $A$ ikke har invers.

\oppgaveDel{c}

\lstinputlisting[language=Python]{5c.py}

Ettersom $x_1, x_2, x_3, x_4$ representerer antall produserte blandinger må verdiene være $0$ eller positivt heltall.\\
Løsningen forteller oss at $x_3$ er en fri variabel, i tillegg til at $x_3 >= 8$ for å hindre at $x_2$ blir negativ, og $x_3 <= 26$ for å hindre at $x_1$ blir negativ.\\
Dette gir $x_1 = 56 - 2 t, \ \ x_2 = t - 8, \ \ x_3 = t, \ \ x_4 = 16, \ \ t \in \{ 8 \idots 26 \}.$

\oppgaveDelSlutt

% % % % % % % % % % % % % % % % % % % % % % % % % % % % % % % % % % % % % % % % 

\end{document}

