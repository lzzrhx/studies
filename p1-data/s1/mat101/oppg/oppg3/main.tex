\documentclass[norsk,8pt,a4paper]{report}
\usepackage[margin=1.5cm,tmargin=1.0cm,bmargin=1.0cm,rmargin=1.5cm,lmargin=1.5cm,footskip=0.2cm]{geometry}
\title{MAT101 - Innlevering 3 (Gruppe 16)}
\author{}
\date{}

% % % % % % % % % % % % % % % % % % % % % % % % % % % % % % % % % % % % % % % % 

\usepackage{babel}               % Support for other languages than english
\usepackage{setspace}            % Set paragraph spacing
\usepackage{nicefrac}            % Nice looking fractals
\usepackage{graphicx}            % Images
\usepackage{gensymb}             % Degree symbol
\usepackage{listings}            % Code
\usepackage{ulem}                % Double underline
\usepackage{amssymb}             % ...
%\usepackage{pdfpages}            % Insert pdf pages
\usepackage{enumitem}            % Lists
\usepackage{colortbl}            % Colored tables
\usepackage[compact]{titlesec}   % ...
\usepackage[T1]{fontenc}         % ...
\usepackage[utf8]{inputenc}      % ...
\usepackage[fleqn]{amsmath}      % ...
\usepackage[makeroom]{cancel}    % ...
%\usepackage{empheq}              % ...
%\setstretch{1.5}
\setlength{\parindent}{0pt}
\titlespacing*{\subsection}{0cm}{1.5cm}{0.5cm}
\lstset{
aboveskip=0cm,
belowskip=0cm,
showstringspaces=false,
columns=flexible,
basicstyle={\scriptsize\ttfamily},
breaklines=true,
breakatwhitespace=true,
tabsize=4,
inputencoding = utf8,  % Input encoding
extendedchars = true,  % Extended ASCII
literate      =        % Support additional characters
{⋆}{{$\star\ $}}1 {∘}{{$\circ\ $}}1
{á}{{\'a}}1  {é}{{\'e}}1  {í}{{\'i}}1 {ó}{{\'o}}1  {ú}{{\'u}}1
{Á}{{\'A}}1  {É}{{\'E}}1  {Í}{{\'I}}1 {Ó}{{\'O}}1  {Ú}{{\'U}}1
{à}{{\`a}}1  {è}{{\`e}}1  {ì}{{\`i}}1 {ò}{{\`o}}1  {ù}{{\`u}}1
{À}{{\`A}}1  {È}{{\`E}}1  {Ì}{{\`I}}1 {Ò}{{\`O}}1  {Ù}{{\`U}}1
{ä}{{\"a}}1  {ë}{{\"e}}1  {ï}{{\"i}}1 {ö}{{\"o}}1  {ü}{{\"u}}1
{Ä}{{\"A}}1  {Ë}{{\"E}}1  {Ï}{{\"I}}1 {Ö}{{\"O}}1  {Ü}{{\"U}}1
{â}{{\^a}}1  {ê}{{\^e}}1  {î}{{\^i}}1 {ô}{{\^o}}1  {û}{{\^u}}1
{Â}{{\^A}}1  {Ê}{{\^E}}1  {Î}{{\^I}}1 {Ô}{{\^O}}1  {Û}{{\^U}}1
{œ}{{\oe}}1  {Œ}{{\OE}}1  {æ}{{\ae}}1 {Æ}{{\AE}}1  {ß}{{\ss}}1
{ẞ}{{\SS}}1  {ç}{{\c{c}}}1 {Ç}{{\c{C}}}1 {ø}{{\o}}1  {Ø}{{\O}}1
{å}{{\aa\ }}1  {Å}{{\AA}}1  {ã}{{\~a}}1  {õ}{{\~o}}1 {Ã}{{\~A}}1
{Õ}{{\~O}}1  {ñ}{{\~n}}1  {Ñ}{{\~N}}1  {¿}{{?`}}1  {¡}{{!`}}1
{„}{\quotedblbase}1 {“}{\textquotedblleft}1 {–}{$-$}1
{°}{{\textdegree}}1 {º}{{\textordmasculine}}1 {ª}{{\textordfeminine}}1
{£}{{\pounds}}1  {©}{{\copyright}}1  {®}{{\textregistered}}1
{«}{{\guillemotleft}}1  {»}{{\guillemotright}}1  {Ð}{{\DH}}1  {ð}{{\dh}}1
{Ý}{{\'Y}}1    {ý}{{\'y}}1    {Þ}{{\TH}}1    {þ}{{\th}}1    {Ă}{{\u{A}}}1
{ă}{{\u{a}}}1  {Ą}{{\k{A}}}1  {ą}{{\k{a}}}1  {Ć}{{\'C}}1    {ć}{{\'c}}1
{Č}{{\v{C}}}1  {č}{{\v{c}}}1  {Ď}{{\v{D}}}1  {ď}{{\v{d}}}1  {Đ}{{\DJ}}1
{đ}{{\dj}}1    {Ė}{{\.{E}}}1  {ė}{{\.{e}}}1  {Ę}{{\k{E}}}1  {ę}{{\k{e}}}1
{Ě}{{\v{E}}}1  {ě}{{\v{e}}}1  {Ğ}{{\u{G}}}1  {ğ}{{\u{g}}}1  {Ĩ}{{\~I}}1
{ĩ}{{\~\i}}1   {Į}{{\k{I}}}1  {į}{{\k{i}}}1  {İ}{{\.{I}}}1  {ı}{{\i}}1
{Ĺ}{{\'L}}1    {ĺ}{{\'l}}1    {Ľ}{{\v{L}}}1  {ľ}{{\v{l}}}1  {Ł}{{\L{}}}1
{ł}{{\l{}}}1   {Ń}{{\'N}}1    {ń}{{\'n}}1    {Ň}{{\v{N}}}1  {ň}{{\v{n}}}1
{Ő}{{\H{O}}}1  {ő}{{\H{o}}}1  {Ŕ}{{\'{R}}}1  {ŕ}{{\'{r}}}1  {Ř}{{\v{R}}}1
{ř}{{\v{r}}}1  {Ś}{{\'S}}1    {ś}{{\'s}}1    {Ş}{{\c{S}}}1  {ş}{{\c{s}}}1
{Š}{{\v{S}}}1  {š}{{\v{s}}}1  {Ť}{{\v{T}}}1  {ť}{{\v{t}}}1  {Ũ}{{\~U}}1
{ũ}{{\~u}}1    {Ū}{{\={U}}}1  {ū}{{\={u}}}1  {Ů}{{\r{U}}}1  {ů}{{\r{u}}}1
{Ű}{{\H{U}}}1  {ű}{{\H{u}}}1  {Ų}{{\k{U}}}1  {ų}{{\k{u}}}1  {Ź}{{\'Z}}1
{ź}{{\'z}}1    {Ż}{{\.Z}}1    {ż}{{\.z}}1    {Ž}{{\v{Z}}}1  {ž}{{\v{z}}}1
}

\newlength{\myeqskip}  \setlength{\myeqskip}{2pt}

% % % % % % % % % % % % % % % % % % % % % % % % % % % % % % % % % % % % % % % % 

\newcommand{\oppgave}[1]{\subsection*{Oppgave #1}}
\newcommand{\oppgaveDelStart}{\begin{enumerate}[leftmargin=*,itemsep=1.5cm,labelsep=1.5em,label=\alph*)]}
\newcommand{\oppgaveDelSlutt}{\end{enumerate}}
\newcommand{\oppgaveDel}[1]{\item[#1)]}

% % % % % % % % % % % % % % % % % % % % % % % % % % % % % % % % % % % % % % % % 

\begin{document}

% % % % % % % % % % % % % % % % % % % % % % % % % % % % % % % % % % % % % % % % 

\oppgave{1}
\oppgaveDelStart

\oppgaveDel{a}
\begin{align*}
&\sum\limits_{k = 1}^{100} k \cdot 2^k\\
\end{align*}

\oppgaveDel{b}
\begin{align*}
&\text{Basis (1):}\\
&F_1 = 1 \cdot 2^1 = 2\\\\
&Rekursjon:\\
&F_n = F_{n-1} + n \cdot 2^n
\end{align*}

\oppgaveDel{c}
\begin{align*}
&\text{Bevis at }F_n = (n-1)2^{n+1} + 2 \ \text{ for alle } \ n \ge 1\\\\
&\text{La } \ S(n): \sum\limits_{i = 1}^{n} i \cdot 2^i = (n-1)2^{n+1} + 2\\\\
&\text{Basissteget ($1$):}\\
&2 = (1-1)2^{1+1} + 2 &\text{[ fra definisjonen av $F_1$ ]}\\
&\text{HS.: } (1-1)2^{1+1} + 2 = (0)2^2 + 2 = 2\\
&\text{VS. $=$ HS.} \ \ \checkmark\\\\
&\text{Induksjonssteget (k):}\\
&\text{Anta at $S(n)$ holder for $n = k$}\\
&\sum\limits_{i = 1}^{k} i \cdot 2^i = (k-1)2^{k+1} + 2\\\\
&\text{Bevissteget (k + 1):}\\
&\text{For $n = k+1$}\\
&\sum\limits_{i = 1}^{k+1} i \cdot 2^i = \sum\limits_{i = 1}^{k} i \cdot 2^i + (k+1) \cdot 2^{k+1}\\
&\text{HS.: } \sum\limits_{i = 1}^{k} i \cdot 2^i + (k+1) \cdot 2^{k+1} = (k-1)2^{k+1} + 2 + (k+1) \cdot 2^{k+1}&\text{[ fra IH ]}\\
&\text{HS.: } = 2^{k+1} \Big[ (k-1) + (k+1) \Big] + 2\\
&\text{HS.: } = 2^{k+1} (2k) + 2\\
&\text{HS.: } = k \cdot 2^{k+2} + 2\\
&\text{HS.: } = \Big( (k+1) - 1 \Big) 2^{(k+1) + 1} + 2\\
&\text{$S(n)$ holder for $n = k + 1$, derfor holder $s(n)$ for alle $n \ge 1$.}\\\\
\end{align*}

\oppgaveDelSlutt

% % % % % % % % % % % % % % % % % % % % % % % % % % % % % % % % % % % % % % % % 

\oppgave{2}
\oppgaveDelStart

\oppgaveDel{a}
\begin{lstlisting}[language=Haskell]
factorial :: Int -> Int
factorial 0 = 1
factorial n = n * factorial (n - 1)
\end{lstlisting}

\oppgaveDel{b}
\begin{align*}
&\text{Basis (1):}\\
&T_1 = 1\\\\
&Rekursjon:\\
&T_n = T_{n-1} + n \ \text{ for } \ n > 1
\end{align*}

\oppgaveDel{c}
\begin{lstlisting}[language=Haskell]
triangular :: Int -> Int
triangular 1 = 1
triangular n = triangular (n - 1) + n
\end{lstlisting}

\oppgaveDelSlutt

% % % % % % % % % % % % % % % % % % % % % % % % % % % % % % % % % % % % % % % % 

\oppgave{3}
\oppgaveDelStart

\oppgaveDel{a}
\begin{align*}
g(0101) &= g(010) + 3\\
&= g(01) + 0 + 3\\
&= g(0) + 3 + 0 + 3\\
&= 0 + 3 + 0 + 3\\
&= 3 + 3\\
&= 6\\
\end{align*}

\oppgaveDel{b}
Funksjonen $g$ tar antallet $1$'ere i den gitte strengen og multipliserer antallet med 3.

\oppgaveDel{c}
\begin{align*}
&\text{La $P(b)$ være: } g(b) = \frac{1}{3} g(f(b))\\\\
&\text{Og $B$ være settet av alle bitstrenger.}\\\\
\end{align*}
\begin{align*}
&\text{Basissteget:}\\
&1)\\
&b = 0
\begin{cases}
g(0) = 0,\\
f(0) = 0
\end{cases}\\\\
&(1) \ \ g(b) = g(0) = 0\\
&(2) \ \ \frac{1}{3}g(f(b)) = \frac{1}{3}g(f(0)) = \frac{1}{3}g(0) = \frac{1}{3}0 = 0\\\\
&(1) = (2), \ \ \text{$P(b)$ holder.}\\\\
&2)\\
&b = 1
\begin{cases}
g(1) = 3,\\
f(1) = 111
\end{cases}\\\\
&(3) \ \ g(b) = g(1) = 3\\
&(4) \ \ \frac{1}{3}g(f(b)) = \frac{1}{3}g(f(1)) = \frac{1}{3}g(111) = \frac{1}{3}9 = 3\\\\
&(3) = (4), \ \ \text{$P(b)$ holder.}\\\\
\end{align*}
\begin{align*}
&\text{Induksjonshypotesen (IH):}\\
&\text{Anta at $P(b): g(b) = \frac{1}{3} g(f(b))$ holder for en $b \in B$}\\\\
\end{align*}
\begin{align*}
&\text{Bevissteget:}\\
&\text{a) }\\
&b = b0
\begin{cases}
g(b0) = g(b),\\
f(b0) = f(b)
\end{cases}\\\\
&\Rightarrow \frac{1}{3}g(f(b0)) = \frac{1}{3}g(f(b)) = g(b) = g(b0)&\text{[ fra IH og definisjonen av $f(b0)$ og $g(b0)$ ]}\\\\
%&g(b0) = \frac{1}{3} g(f(b0))\\\\
%&\text{HS.: } \ \frac{1}{3}g(f(b0)) = \frac{1}{3}g(f(b)) = g(b) = g(b0)&\text{[ fra IH og definisjonen av $g(b0)$]}\\
%&\text{HS. = VS. } \checkmark\\\\
&\text{b) }\\
&b = b1
\begin{cases}
g(b1) = g(b) + 3,\\
f(b1) = f(b)111
\end{cases}\\\\
&\Rightarrow g(f(b1)) = g(f(b)111) = g(f(b)) + 3 + 3 + 3 = g(f(b)) + 9&\text{[ fra definisjonen av $f(b1)$ og $g(b1)$ ]}\\
&\Rightarrow \frac{1}{3}g(f(b1)) = \frac{1}{3}(g(f(b)) + 9) = \frac{1}{3}g(f(b)) + \frac{1}{3}9 = \frac{1}{3}g(f(b)) + 3\\
&\Rightarrow \frac{1}{3}g(f(b)) + 3 = g(b) + 3 = g(b1) &\text{[ fra IH ]}\\\\\\
&\text{Ved strukturell induksjon er det beist at $P(b)$ holder for $\forall \ b \in B$}\\
\end{align*}

\oppgaveDelSlutt

% % % % % % % % % % % % % % % % % % % % % % % % % % % % % % % % % % % % % % % % 

\oppgave{4}
\oppgaveDelStart

\oppgaveDel{a}
\begin{align*}
&\text{Bevis at $5^n$ -1 er delelig med $4$ for alle $n \ge0$}\\\\
&\text{Basissteget ($n=0$):}\\
&5^0-1 = 0, \text{ som er delelig med $4$}\\\\
&\text{Induksjonshypotesen (IH):}\\
&\text{Anta at $5^k-1$ er delelig med 4}\\
&\Rightarrow5^k-1 = 4m, \ \ m \in \mathbb{N}\\
&\Rightarrow 5^k = 4m + 1, \ \ m \in \mathbb{N}\\\\
&\text{Bevissteget ($n = k+1$):}
\end{align*}
\begin{align*}
5^{k+1} &= 5^1 \cdot 5^k - 1\\
&= 5(4m + 1) - 1&\text{[ fra IH ]}\\
&= 20m + 5 - 1\\
&= 4(5m+1)\\\\
&4(5m+1) \ \text{ der $m \in \mathbb{N}$ er delelig med $4$ } \checkmark
\end{align*}

\oppgaveDel{b}
\begin{align*}
&\text{Bevis at $ \ 2^n < (n+1)! \ $ for alle $n \ge 2$}\\\\
&\text{Basissteget ($n=2$):}\\
&2^2 = 4 \ \text{ og } \ (2+1)! = 6\\
&\Rightarrow 4 < 6, \text{ det holder} \ \checkmark\\\\
&\text{Induksjonshypotesen (IH):}\\
&\text{Anta at $2^k < (k+1)!$}\\\\
&\text{Bevissteget ($n = k+1$):}
\end{align*}
\begin{align*}
2^{k+1} &= 2^1 \cdot 2^k\\
\Rightarrow 2 \cdot 2^k &< 2 \cdot (k+1)!&\text{[ fra IH ]}\\
&<(k+2) \cdot (k+1)!\\
%&\\
&= (k+2)! = ((k+1)+1))!\\
\end{align*}
\begin{align*}
&\text{Derfor er } \ 2^{k+1} < ((k+1)+1)! \ \text{ og $ \ 2^n < (n+1)! \ $ holder for alle $n \ge 2$.}\\
\end{align*}

\oppgaveDelSlutt

% % % % % % % % % % % % % % % % % % % % % % % % % % % % % % % % % % % % % % % % 

\oppgave{5}
\oppgaveDelStart

\oppgaveDel{a}
\begin{align*}
&S = \{ (1, 3),  (2, 6), (3,9), (4,12), (5,15)\}\\
\end{align*}

\oppgaveDel{b}
\begin{align*}
&\text{La $P( \ \langle x,y\rangle \ )$ være: } x+y = \text{ $0$ (mod $4$) }\\\\
&\text{Basissteg $\langle 1, 3 \rangle$:}\\
&1+3 = 4 \equiv 0 \pmod 4\\
&\text{Det holder } \checkmark\\\\
&\text{Induksjonshypotesen (IH):}\\
&\text{Anta at $P( \ \langle x,y\rangle \ )$ holder: } x+y \equiv 0 \pmod 4\\\\
&\text{Bevissteget $P( \ \langle x+1,y+3\rangle \ )$:}\\
&(x+1) + (y+3) = x+y + 4 \equiv 0 + 0 \equiv \text{ $0$ (mod $4$) }\\\\
&x + y \equiv \text{ $0$ (mod $4$) } \forall \langle x,y \rangle \in S \text{ er bevist ved strukturell induksjon.}\\
\end{align*}

\oppgaveDelSlutt

% % % % % % % % % % % % % % % % % % % % % % % % % % % % % % % % % % % % % % % % 

\oppgave{6}
\oppgaveDelStart

\oppgaveDel{a}
\begin{align*}
113 &= 0111 \ 0001_2\\
-27 &= 1110 \ 0101_2\\
\end{align*}

\oppgaveDel{b}
\begin{align*}
&0100 \ 1011_2 = 2^6 + 2^3 + 2^1 + 2^0 = 64 + 8 + 2 + 1 = 75\\
&1110 \ 0100_2 = -2^7 + 2^6 + 2^5 + 2^2 = -128 + 64 + 32 + 4 = -28\\
\end{align*}

\oppgaveDelSlutt

% % % % % % % % % % % % % % % % % % % % % % % % % % % % % % % % % % % % % % % % 

\oppgave{7}
\oppgaveDelStart

\oppgaveDel{a}
\begin{align*}
&0111_2 = 2^2 + 2^1 + 2^0 = 4 + 2 + 1 = 7\\
\end{align*}

\oppgaveDel{b}
\begin{align*}
&0111_2 + 0010_2 = 1001_2 = -2^4 + 2^0 = -8 + 1 = -7\\
\end{align*}

\oppgaveDel{c}
\begin{align*}
&3 = 2 + 1 = 2^1 + 2^0 = 0011_2\\
&-5 = -8 + 2 + 1 = -2^3 + 2^1 + 2^0 = 1011_2\\
&0011_2 + 1011_2 = 1110_2 = -2^3 + 2^2 + 2^1 = -8 + 4 + 2 = -2
\end{align*}

\oppgaveDelSlutt

% % % % % % % % % % % % % % % % % % % % % % % % % % % % % % % % % % % % % % % % 

\end{document}
