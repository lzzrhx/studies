\documentclass{report}%\documentclass[11pt,a4paper,norsk]{report}

\title{REAL112-1 24H - Matematikk\\Obligatorisk innlevering 3}
\author{Casper Eide Özdemir-Børretzen}
\date{}

\makeatletter
\newcommand{\institle}{\@title}
\newcommand{\insauthor}{\@author}
\newcommand{\insdate}{\@date}
\makeatother

\input{preamble}
\input{macros}
\input{letterfonts}

\usepackage{tikz}
\usetikzlibrary{positioning}

\usepackage{pdfpages}
\usepackage{ulem}
\usepackage[utf8]{inputenc}
\usepackage[T1]{fontenc}
\usepackage{textcomp}
\usepackage{url}
\usepackage{hyperref}
\usepackage[norsk]{babel}
\usepackage{csquotes}
\usepackage{graphicx}
\hypersetup{urlcolor=blue,pdftitle={\institle},pdfauthor={\insauthor}}
\urlstyle{sf}

\begin{document}
\includepdf[pages={1}]{forside.pdf}
%\maketitle
%\newpage



% OPPGAVE 1
\oppg{}{
a)\\
\hspace*{8mm}Fellesnevner er \uuline{$x^2 + 2x$}\\

\vspace{0.25cm}

b)
\begin{align*}
     &\frac{2}{x+2} + \frac{1}{x^2 + 2x} + \frac{3}{2}\\\\
=\ \ &\frac{(2)(x)}{(x + 2)(x)} + \frac{1}{x^2 + 2} + \frac{(3)(\frac{1}{2}x^2 + x)}{(2)(\frac{1}{2}x^2 + x)}\\\\
=\ \ &\frac{2x}{x^2 + 2x} + \frac{1}{x^2 + 2x} + \frac{\frac{3}{2}x^2 + 3x}{x^2 + 2x}\\\\
=\ \ &\frac{2x + 1 + \frac{3}{2}x^2 + 3x}{x^2 + 2x}\\\\
=\ \ &\frac{\frac{3}{2}x^2 + 5x + 1}{x^2 + 2x}\\
\end{align*}
}

\vspace{0.75cm}



% OPPGAVE 2
\oppg{}{
a)\\
\hspace*{8mm}i)\\
\hspace*{14mm}$x^4 - 3x + 2$\\
\hspace*{14mm}Dette er et fjerdegrads polynom.\\

\hspace*{8mm}ii)\\
\hspace*{14mm}$-x^3 + 3x^2 + 3$\\
\hspace*{14mm}Dette er et tredjegrads polynom.\\

\hspace*{8mm}iii)\\
\hspace*{14mm}$x^2 - 3x + 3$\\
\hspace*{14mm}Dette er et andregrads polynom.\\

\vspace{0.25cm}

b)\\
\hspace*{8mm}A: Ekstremalpunkt, bunnpunkt\\
\hspace*{8mm}B: Ekstremalpunkt, toppunkt\\
\hspace*{8mm}C: Ekstremalpunkt, bunnpunkt\\
\hspace*{8mm}D: Nullpunkt\\
\hspace*{8mm}E: Nullpunkt\\
\hspace*{8mm}F: Nullpunkt\\
\hspace*{8mm}G: Nullpunkt\\

\hspace*{8mm}Dette er en fjerdegrads polynomfunksjon.\\
}

\vspace{0.75cm}



% OPPGAVE 3
\oppg{}{
a)
\begin{equation*}
\begin{aligned}
(x^2 - 4x &- 5) : (x - 1) = x - 3 + \frac{- 8}{x - 1}\\
-(x^2 - x &+ 0)\\
\noalign{\smallskip} \hline \noalign{\smallskip}
  -3x &- 5\\
-(-3x &+ 3)\\
\noalign{\smallskip} \hline \noalign{\smallskip}
&-8
\end{aligned}
\end{equation*}

\vspace{0.25cm}

b)\\
\hspace*{8mm}$P(x) = x^2 - 4x - 5$\\
\hspace*{8mm}Divisjonen $P(x) : (x + 2)$ vil gå opp dersom $P(-2) = 0$.\\
\hspace*{8mm}$P(-2) = (-2)^2 - 4(-2) - 5 = \uuline{7}$\\
\hspace*{8mm}\uuline{Divisjonen vil ikke gå opp og vil få 7 i rest.}\\

\vspace{0.25cm}

c)\\
\begin{align*}
%(x^3 + ax^2 - x + 2) : (x + 2)\\
P(x) &= x^3 + ax^2 - x + 2\\
P(x) : (x + 2) &= 0\\
P(-2) &= 0\\
(-2)^3 + a(-2)^2 - (-2) + 2 &= 0\\
-8 - 4a + 2 + 2 &= 0\\
-4a &= 8 - 2 - 2\\
\frac{-4a}{-4} &= \frac{4}{-4}\\
a &= \uuline{-1}\\
\end{align*}
}

%\vspace{0.75cm}
\newpage



% OPPGAVE 4
\oppg{}{
a)\\
\hspace*{8mm}$\frac{x^2 - 4x - 5}{x + 2}>0$\\

\hspace*{8mm}Ettersom $1 + (-5) = -4$ og $1 \times (-5) = -5$ er\\

\hspace*{8mm}$(x^2 - 4x - 5) = (x + 1)(x - 5)$\\

\hspace*{8mm}og ulikheten kan skrives som\\

\hspace*{8mm}$(x + 1)(x - 5) : (x + 2) > 0$\\

\hspace*{8mm}Dette kan legges inn i et fortegnsskjema:\\

\begin{tikzpicture}[%
negativ/.style={blue,dashed},
negativ_gray/.style={gray,dashed},
positiv/.style={blue},
positiv_gray/.style={gray},
vertlinje/.style={dotted,opacity=.7},
node distance=1.5ex,
nullpunkt/.style={fill=white,inner sep= 3pt}]
 \draw [->,>=stealth] (-4,0) node (linestart) {} -- (7,0) node (lineend) {};
 \node (null1) at (-1,0) [label=above:-1] {};
 \node (null2) at (5,0) [label=above:5] {};
 \node (null3) at (-2,0) [label=above:-2] {};
 \node [matrix] (produktledd) [below left=of linestart]{
   \node [left] (f1) {$x+1$}; \\
   \node [left] (f2) {$x-5$}; \\
   \node [left] (f3) {$x+2$}; \\
       \node [left] (f)  {$\frac{x^2 - 4x - 5}{x + 2}$}; \\
  };
 \draw [vertlinje] (null1)       -- (null1 |- f);
 \draw [vertlinje] (null2)       -- (null2 |- f);
 \draw [vertlinje] (null3)       -- (null3 |- f);
 
 \draw [positiv_gray]   (null1 |- f1) -- (lineend |- f1);
 \draw [negativ_gray]   (f1)          -- (null1 |- f1) node[nullpunkt] {$0$};
 
 \draw [positiv_gray]   (null2 |- f2) -- (lineend |- f2);
 \draw [negativ_gray]   (f2)          -- (null2 |- f2) node[nullpunkt] {$0$};
 
 \draw [positiv_gray]   (null3 |- f3) -- (lineend |- f3);
 \draw [negativ_gray]   (f3)          -- (null3 |- f3) node[nullpunkt] {$0$};
 
 \draw [negativ]   (null1 |- f)  -- (null2 |- f);
 \draw [positiv]   (null3 |- f) -- (null1 |- f) node[nullpunkt] {$0$};
 \draw [positiv]   (null2 |- f) node[nullpunkt] {$0$} -- (lineend |- f);
 \draw [negativ]   (f)           -- (null3 |- f)  node[nullpunkt] {$><$}
 %                  (null2 |- f) node[nullpunkt] {$0$} -- (lineend |- f);
\end{tikzpicture}

\vspace{0.25cm}

\hspace*{8mm}$\frac{x^2 - 4x - 5}{x + 2} > 0$ når \uuline{$-2<x<-1$} og når \uuline{$5 < x$}.\\

\vspace{0.25cm}

b)\\

\hspace*{8mm}$x - 1 < x^2 - 3x + 2 < x + 2$\\

\hspace*{8mm}Denne doble ulikheten kan løses grafisk:

\vspace{0.25cm}

\hspace*{8mm}\includegraphics[scale=0.3]{4b.png}

\vspace{0.1cm}
}

\vspace{0.75cm}

% OPPGAVE 5
\oppg{}{
$$\lim_{x \to 3} \frac{x - 3}{x + 3}$$
\begin{center}Først kan vi prøve med x = 3 for å se om det gir et annet resultat enn $\frac{0}{0}$.\end{center}
$$\frac{3-3}{3+6} = \frac{0}{6} = \uuline{0}$$

\vspace{1cm}

$$
\lim_{x \to \infty} \left( 2 + \frac{x}{x^2 + 1} \right) = \lim_{x \to \infty} \left( 2 + \frac{(x) \times \frac{1}{x^2}}{(x^2 + 1) \times \frac{1}{x^2}} \right) = \lim_{x \to \infty} \left( 2 + \frac{\frac{1}{x}}{1 + \frac{1}{x^2}} \right) = 2 + \frac{0}{ 1 + 0} = \uuline{2}
$$

\vspace{1cm}

$$\lim_{x \to 2} \frac{x^2 - 4}{x - 2}$$
\begin{center}Først kan vi prøve med x = 2 for å se om det gir et annet resultat enn $\frac{0}{0}$.\end{center}
$$\frac{2^2 - 4}{2 - 2} = \frac{0}{0}$$
\begin{center}Det gikk ikke. Da faktoriserer vi teller og nevner med mål om å fjerne $(x - 2)$ for å unngå multiplikasjon med 0.\end{center}
$$\lim_{x \to 2} \frac{x^2 - 4}{x - 2} = \lim_{x \to 2} \frac{\cancel{(x - 2)}(x + 2)}{\cancel{(x - 2)}} = 2 + 2 = \uuline{4}$$
}

%\vspace{0.75cm}
\newpage

% OPPGAVE 6
\oppg{}{
a)
$$f(x) = x^2 - 3x + 2$$
\begin{center}
\uuline{$f$ er en polynomfunksjon og alle polynomfunksjoner er kontinuerlige.}
\end{center}\\
\vspace{0.5cm}
b)
$$g(x) = \frac{x^2 - 1}{x + 2}$$
\begin{center}
En kontinuerlig funksjon har aldri null i nevner.\\
\end{center}
$$g(-2) = \frac{(-2)^2 - 1}{(-2) + 2} = \frac{3}{0}$$
\begin{center}
\uuline{Funksjonen $g$ har null i nevner ved $x = -2$ og er ikke en kontinuerlig funksjon.}
\end{center}
\vspace{0.5cm}
c)
$$h(x) = \frac{x^2 - 1}{x + 1}$$
\begin{center}
En kontinuerlig funksjon har aldri null i nevner.\\
\end{center}
$$h(-1) = \frac{(-1)^2 - 1}{(-1) + 1} = \frac{0}{0}$$
\begin{center}
\uuline{Funksjonen $h$ har null i nevner ved $x = -1$ og er ikke en kontinuerlig funksjon.}
\end{center}
\vspace{0.5cm}
d)
$$
p(x)=
\begin{cases}
2x - 1, &x \leq 2\\
2 - x, &x >2
\end{cases}
$$
\begin{alignat*}{2}
&\lim_{x \to 2^-} p(x) = \lim_{x \to 2^-} (2x -1 ) = 2 \times 2 - 1 \ &&= 3\\\\
&\lim_{x \to 2^+} p(x) = \lim_{x \to 2^+} (2 - x ) = 2 - 2 \ &&= 0
\end{alignat*}
\begin{center}
\uuline{Funksjonen $p$ er ikke kontinuerlig for $x = 2$.}
\end{center}
\vspace{0.1cm}
}

% OPPGAVE 7
\newpage
\oppg{}
{
a)
$$f(x) = \frac{x - 2}{x + 3}$$
$$f(-3) = \frac{(-3)-2}{(-3)+3} = \frac{-5}{0}$$
\begin{center}Ved $x = -3$ får vi $0$ i nevneren og en negativ verdi i telleren.\end{center}
\begin{center}$f(x)\rightarrow\infty$ når $x \rightarrow -3$\end{center}
\begin{center}Fuksjonen har en vertikal asymptote ved $x = -3$\end{center}
\vspace{0.4cm}
b)
\begin{center}Hvis $x$ verdien går mot et tall $a$ som gir $=0$ i nevner og $\neq0$ i teller har grafen en vertikal asymptote.\end{center}
\begin{center}Her får vi at $x \rightarrow a$ gir $|f(x)| \rightarrow \infty$\end{center}
\vspace{0.4cm}
c)
$$\lim_{|x| \rightarrow \infty} f(x) = 1$$
\begin{center}Grafen har en horisontal asymptote ved $y = 1$\end{center}
\vspace{0.4cm}
d)
\begin{center}Nullpunktet i $y$ aksen finner vi når teller $=0$ og nevner $\neq 0$\end{center}
$$f(2) = \frac{2 - 2}{2 + 3} = \frac{0}{5}$$
\vspace{0.1cm}
\begin{center}Nullpunktet i $x$ aksen finner vi når $x=0$\end{center}
$$f(0) = \frac{0-2}{0+3} = \frac{-2}{3}$$
\vspace{0.4cm}
e)\\
\begin{center}\includegraphics[scale=0.3]{7e.png}\end{center}
\vspace{0.1cm}
}

% OPPGAVE 8
\newpage
\oppg{}
{
$$f(x) &= x^2 - x - 2$$
\begin{center}\includegraphics[scale=0.3]{8.png}\end{center}
\vspace{0.4cm}
a)
$$\frac{\Delta y}{\Delta x} = \frac{f(4) - f(2)}{4 - 2} = \frac{10 - 0}{4-2} = \frac{10}{4} = 2.5$$
\vspace{0.4cm}
b)
$$\frac{\Delta y}{\Delta x} = \frac{f(3) - f(2)}{3 - 2} = \frac{4 - 0}{3 - 2} = \frac{4}{1} = 4$$
\vspace{0.4cm}
c)
\begin{align*}
f(x + \Delta x) &= (x + \Delta x)^2 - (x + \Delta x) - 2\\\\
f(2 + \Delta x) &= (2 + \Delta x)^2 - (2 + \Delta x) - 2\\\\
&=\ 4 + 4 \Delta x + (\Delta x)^2 - 2 - \Delta x - 2\\\\
&=\ (\Delta x)^2 + 3 \Delta x\\\\
\lim_{\Delta x \to 0} \frac{f(2 + \Delta x) - f(2)}{\Delta x} &= \lim_{\Delta x \to 0} \frac{(\Delta x)^2 + 3 \Delta x}{\Delta x} = \lim_{\Delta x \to 0} \frac{\cancel{\Delta x} (\Delta x + 3)}{\cancel{\Delta x}}\\\\
&=\ \lim_{\Delta x \to 0}(\Delta x + 3) = 0 + 3 = \uuline{3}\\\\
\end{align*}
\vspace{0.4cm}
d)
$$y = 3x - 6$$
\vspace{0.1cm}
}

\end{document}
