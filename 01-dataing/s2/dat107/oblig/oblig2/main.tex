%! TeX program = lualatex
% - - - - - - - - - - - - - - - - - - - - - - - - - - - - - - - - - - - - - - - 
% Core setup
% - - - - - - - - - - - - - - - - - - - - - - - - - - - - - - - - - - - - - - - 
\documentclass[10pt,a4paper]{article}
\usepackage[tmargin=2.5cm,rmargin=2.5cm,bmargin=2.5cm,lmargin=2.5cm]{geometry}
\usepackage[norsk]{babel}         % Support for other languages than english
\usepackage[utf8]{inputenc}       % Input encoding
\usepackage[T1]{fontenc}          % Font encoding
\usepackage{microtype}            % Improve text appearance in numerous ways
\usepackage{textcomp}             % Add extra symbols
\usepackage{mathtools}            % Improvements for math
\usepackage{amssymb}              % Extended math symbols


% - - - - - - - - - - - - - - - - - - - - - - - - - - - - - - - - - - - - - - - 
% Font packages
% - - - - - - - - - - - - - - - - - - - - - - - - - - - - - - - - - - - - - - - 
%\usepackage[p,osf]{fbb}          % Main font
%\usepackage[scaled=.95,type1]{cabin} % Sans-serif font
%\usepackage[libertine]{newtxmath} % Math font
\usepackage[osf]{newpxtext}       % Main (serif/sans-serif) font
\usepackage{newpxmath}            % Math font
\usepackage[scaled]{beramono}     % Monospace font
\usepackage{iftex}
\ifLuaTeX
    \usepackage{fontspec}                % Custom fonts
    \setmonofont{PxPlus ToshibaSat 8x16} % Override monospace font
\fi


% - - - - - - - - - - - - - - - - - - - - - - - - - - - - - - - - - - - - - - - 
% Additional packages
% - - - - - - - - - - - - - - - - - - - - - - - - - - - - - - - - - - - - - - - 
\usepackage{soul}                 % Character spacing
\usepackage[T1]{url}              % Clickable URLs
\usepackage{hyperref}             % Clickable references
\usepackage{csquotes}             % Support quote marks in various languages
\usepackage{ulem}                 % Double underline
\usepackage[makeroom]{cancel}     % Crossing out text
\usepackage{nicefrac}             % Nice looking fractals
\usepackage{gensymb}              % Degree symbol
\usepackage[compact]{titlesec}    % Title style
\usepackage{setspace}             % Set paragraph spacing
\usepackage{xcolor}               % Colors
\usepackage{enumitem}             % Lists
\usepackage{listings}             % Insert code
\usepackage{tikz}                 % Drawing of graphics
\usepackage{graphicx}             % Insert images
\usepackage{pdfpages}             % Insert pdf pages
\usepackage{biblatex}             % Bibliography


% - - - - - - - - - - - - - - - - - - - - - - - - - - - - - - - - - - - - - - - 
% Include files
% - - - - - - - - - - - - - - - - - - - - - - - - - - - - - - - - - - - - - - - 
% - - - - - - - - - - - - - - - - - - - - - - - - - - - - - - - - - - - - - - - 
% Kråkefot-notasjon
% Basert på kode fra: https://tex.stackexchange.com/questions/462914/how-to-create-an-er-diagram-using-tikzpicture-environment
% - - - - - - - - - - - - - - - - - - - - - - - - - - - - - - - - - - - - - - - 
\usepackage{array}
\pgfdeclarearrow{
    name = mmany,
    parameters = {}, setup code = {}, defaults = {},
    drawing code={
        \newdimen\arrowsize%
        \arrowsize=0.5pt%
        \pgfsetdash{}{+0pt}%
        \pgfsetmiterjoin%
        \advance\arrowsize by .25\pgflinewidth%
        \pgfpathmoveto{\pgfqpoint{0pt}{-9\arrowsize}}%
        \pgfpathlineto{\pgfqpoint{-13\arrowsize}{0pt}}%
        \pgfpathlineto{\pgfqpoint{0pt}{9\arrowsize}}%
        \pgfpathmoveto{\pgfqpoint{-19\arrowsize}{-10\arrowsize}}% 
        \pgfpathlineto{\pgfqpoint{-19\arrowsize}{10\arrowsize}}% 
        \pgfusepathqstroke%
    }
}

\pgfdeclarearrow{
    name = omany, 
    parameters = {}, setup code = {}, defaults = {},
    drawing code = {
        \newdimen\arrowsize%
        \arrowsize=0.5pt%
        \pgfsetdash{}{+0pt}%
        \pgfsetmiterjoin%
        \advance\arrowsize by .25\pgflinewidth%
        \pgfpathmoveto{\pgfqpoint{0pt}{0pt}}%
        \pgfpathlineto{\pgfqpoint{-13\arrowsize}{0pt}}%
        \pgfpathlineto{\pgfqpoint{0pt}{9\arrowsize}}%
        \pgfpathlineto{\pgfqpoint{-13\arrowsize}{0pt}}%
        \pgfpathlineto{\pgfqpoint{0pt}{-9\arrowsize}}%
        \pgfusepathqstroke%
        \pgfsetfillcolor{white}
        \pgfpathcircle{\pgfpoint{-19\arrowsize}{0}} {6\arrowsize}%
        \pgfusepathqfillstroke%
    }
}

\pgfdeclarearrow{
    name = mone,
    parameters = {}, setup code = {}, defaults = {},
    drawing code = {
        \newdimen\arrowsize%
        \arrowsize=0.5pt%%
        \pgfsetdash{}{+0pt}%
        \pgfsetmiterjoin%
        \advance\arrowsize by .25\pgflinewidth%
        \pgfpathmoveto{\pgfqpoint{-9\arrowsize}{-10\arrowsize}}% 
        \pgfpathlineto{\pgfqpoint{-9\arrowsize}{10\arrowsize}}% 
        \pgfpathmoveto{\pgfqpoint{-19\arrowsize}{-10\arrowsize}}% 
        \pgfpathlineto{\pgfqpoint{-19\arrowsize}{10\arrowsize}}%    
        \pgfusepathqstroke%
    }
}

\pgfdeclarearrow{
    name = oone,
    parameters = {}, setup code = {}, defaults = {},
    drawing code = {
        \newdimen\arrowsize%
        \arrowsize=0.5pt%
        \pgfsetdash{}{+0pt}%
        \pgfsetmiterjoin%
        \advance\arrowsize by .25\pgflinewidth
        \pgfpathmoveto{\pgfqpoint{-9\arrowsize}{-10\arrowsize}}% 
        \pgfpathlineto{\pgfqpoint{-9\arrowsize}{10\arrowsize}}% 
        \pgfsetfillcolor{white}
        \pgfpathcircle{\pgfpoint{-19\arrowsize}{0}} {6\arrowsize}%
        \pgfusepathqfillstroke%
    }
}


% Label 
\tikzset{pics/label/.style n args={4}{code={
    \node[at=($(#2)!0.55!(#3)$),label=#4:{\small\ttfamily{#1}},outer sep=0pt,minimum size=0pt]{};
}}}

% Entity
\tikzset{pics/entity/.style n args={3}{code={
    \node[draw,
    %rounded corners=1pt, 
    rectangle split,
    rectangle split parts=2,
    font=\scriptsize\ttfamily\bfseries,
    fill=white,
    drop shadow={opacity=0.25}
    text height=1.5ex,
    ] (#1)
    {\small{#2} \nodepart{second}
    \begin{tabular}{>{\raggedright\arraybackslash}p{4em}>{\raggedright\arraybackslash}p{8.5em}}
    #3
    \end{tabular}};
}}}

%\tikzset{one to one/.style={mone-mone}}
%\tikzset{one to oone/.style={mone-oone}}
%\tikzset{one to many/.style={mone-mmany}}
%\tikzset{one to omany/.style={mone-omany}}
%\tikzset{pics/entitynoatt/.style n args={2}{code={%
%        \node[draw,
%        text height=1.5ex,
%        ] (#1)
%        {#2};%
%    }}
%\tikzset{
%    zig zag/.style={
%        to path={(\tikztostart) -| ($(\tikztostart)!#1!(\tikztotarget)$) |- (\tikztotarget)}
%    },
%\tikzset{
%    zig zag/.default=0.5, 



% - - - - - - - - - - - - - - - - - - - - - - - - - - - - - - - - - - - - - - - 
% Apprentice colorscheme (https://github.com/romainl/Apprentice)
% - - - - - - - - - - - - - - - - - - - - - - - - - - - - - - - - - - - - - - - 
\definecolor{color0}{HTML}{1C1C1C}
\definecolor{color1}{HTML}{AF5F5F}
\definecolor{color2}{HTML}{5F875F}
\definecolor{color3}{HTML}{87875F}
\definecolor{color4}{HTML}{5F87AF}
\definecolor{color5}{HTML}{5F5F87}
\definecolor{color6}{HTML}{5F8787}
\definecolor{color7}{HTML}{6C6C6C}
\definecolor{color8}{HTML}{444444}
\definecolor{color9}{HTML}{FF8700}
\definecolor{color10}{HTML}{87AF87}
\definecolor{color11}{HTML}{FFFFAF}
\definecolor{color12}{HTML}{87AFD7}
\definecolor{color13}{HTML}{8787AF}
\definecolor{color14}{HTML}{5FAFAF}
\definecolor{color15}{HTML}{FFFFFF}
\definecolor{colorfg}{HTML}{BCBCBC}
\definecolor{colorbg}{HTML}{262626}


% - - - - - - - - - - - - - - - - - - - - - - - - - - - - - - - - - - - - - - - 
% Habamax colorscheme (https://github.com/habamax/vim-habamax)
% - - - - - - - - - - - - - - - - - - - - - - - - - - - - - - - - - - - - - - - 
\definecolor{brightCyan}{HTML}{87AFAF}
\definecolor{cyan}{HTML}{5F8787}
\definecolor{darkCyan}{HTML}{1F3F5F}
\definecolor{pink}{HTML}{D75F87}
\definecolor{red}{HTML}{D75F5F}
\definecolor{darkRed}{HTML}{AF5F5F}
\definecolor{brightGreen}{HTML}{5FF75F}
\definecolor{green}{HTML}{87D787}
\definecolor{darkGreen}{HTML}{5FAF5F}
\definecolor{blue}{HTML}{5fafd7}
\definecolor{darkBlue}{HTML}{5F87AF}
\definecolor{brightYellow}{HTML}{ffaf5f}
\definecolor{yellow}{HTML}{d7af87}
\definecolor{darkYellow}{HTML}{af875f}
\definecolor{brightMagenta}{HTML}{ff00af}
\definecolor{magenta}{HTML}{d787d7}
\definecolor{darkMagenta}{HTML}{af87af}


% - - - - - - - - - - - - - - - - - - - - - - - - - - - - - - - - - - - - - - - 
% Custom commands
% - - - - - - - - - - - - - - - - - - - - - - - - - - - - - - - - - - - - - - - 
%\sodef\allcapsspacing{\upshape}{0.15em}{0.65em}{0.6em}
%\sodef\lowsmallcapsspacing{\scshape}{0.075em}{0.5em}{0.6em}
%\newcommand{\allcaps}[1]{\MakeUppercase{\allcapsspacing{#1}}}%   
%\newcommand{\smallcaps}[1]{\MakeLowercase{\textsc{\lowsmallcapsspacing{#1}}}}
\newcommand{\oppgave}[1]{\subsection*{Oppgave #1}}
\newcommand{\oppgaveDelStart}{\begin{enumerate}[leftmargin=*,itemsep=1.5cm,labelsep=1.5em,label=\alph*)]}
\newcommand{\oppgaveDelSlutt}{\end{enumerate}}
\newcommand{\oppgaveDel}[1]{\item[\textbf{#1})]}


% - - - - - - - - - - - - - - - - - - - - - - - - - - - - - - - - - - - - - - - 
% Package settings
% - - - - - - - - - - - - - - - - - - - - - - - - - - - - - - - - - - - - - - - 
\usetikzlibrary{shapes.multipart,shadows}
\usetikzlibrary{positioning}
\usetikzlibrary{calc}
\color{color0}
\linespread{1.05}
\setlength{\parindent}{0pt}
%\setlength{\tabcolsep}{18pt}
%\renewcommand{\arraystretch}{1.25}
\newlength{\myeqskip}\setlength{\myeqskip}{2pt}
\titlespacing*{\section}{0cm}{1.5cm}{0.25cm}
\titlespacing*{\subsection}{0cm}{1cm}{0.5cm}
\titleformat{\section}{\normalfont\large\bfseries}{\thesection}{0.75em}{}
\titleformat{\subsection}{\normalfont\normalsize\bfseries}{\thesubsection.}{0.75em}{}
\titleformat{\subsubsection}{\normalfont\normalsize\itshape}{\thesubsubsection.}{0.75em}{}
\lstset{literate={æ}{{\ae}}1{Æ}{{\AE}}1{ø}{{\o}}1{Ø}{{\O}}1{å}{{\aa}}1{Å}{{\AA}}1,
keywordstyle={\bfseries\color{color1}}, 
commentstyle={\bfseries\color{cyan}},
basicstyle={\scriptsize\ttfamily\color{color0}}, 
numbers=none,numberstyle={\scriptsize\ttfamily\color{color7}},
aboveskip=0.1cm,belowskip=1.0cm,columns=fixed,
breaklines=true,breakatwhitespace=false,keepspaces=true,
showspaces=false,showstringspaces=false,tabsize=4,
captionpos=b,inputencoding=utf8,extendedchars=true}
\tikzstyle{every pic} = [draw=color8]
\tikzstyle{every edge} = [semithick, draw=color8]


% - - - - - - - - - - - - - - - - - - - - - - - - - - - - - - - - - - - - - - - 
% Set title / author
% - - - - - - - - - - - - - - - - - - - - - - - - - - - - - - - - - - - - - - - 
\title{DAT107 \textendash \ Obligatorisk innlevering 2}
\author{\normalsize Gruppemedlemmer: \\ Stephen Neba Fuh, Tord Johan Melheim, \\ Ebubekir Siddik Yuksel, Casper Eide Özdemir-Børretzen}
\date{}


% - - - - - - - - - - - - - - - - - - - - - - - - - - - - - - - - - - - - - - - 
% Document start and title page
% - - - - - - - - - - - - - - - - - - - - - - - - - - - - - - - - - - - - - - - 
\begin{document}
\maketitle


% - - - - - - - - - - - - - - - - - - - - - - - - - - - - - - - - - - - - - - - 
% Todo-list
% - - - - - - - - - - - - - - - - - - - - - - - - - - - - - - - - - - - - - - - 
%\begin{lstlisting}[language=SQL]
%\end{lstlisting}
%\lstinputlisting[language=SQL]{file.sql}
%\oppgave{1}
%\oppgaveDelStart
%\oppgaveDel{a}
%\oppgaveDel{b}
%\oppgaveDel{c}
%\oppgaveDelSlutt

% Huskeliste:
% Primærnøkler
% Fremmednøkler
% Datatyper
% Min/maks kadinalitet
% Sterke/svake entitetstyper og eksistensavhengighet/uavhengighet (kråkefot-notasjon) eller type aggregering (UML-notasjon)
% Redegjørelse/forklaring på at ER-modellen tilfredsstiller hver av 1., 2., og 3. normalform. Tabellene skal faktisk være korrekt normaliserte – det holder ikke å hevde at de er normaliserte dersom de ikke er det.
% Redegjørelse for de valgene som er tatt. Det blir lagt vekt på både praktisk utførelse og teoretisk forståelse
% Besvarelsen skal være “konsistent”, dvs. skal være for den “samme” databasen, og de valgene man har tatt i en del av besvarelsen skal ikke “motsies” av de valgene man har tatt i en annen del av besvarelsen.


% - - - - - - - - - - - - - - - - - - - - - - - - - - - - - - - - - - - - - - - 
% Main content
% - - - - - - - - - - - - - - - - - - - - - - - - - - - - - - - - - - - - - - - 
\begin{center}
\begin{tikzpicture}

% EBok
\pic {entity={A}{EBok}{
CHAR(5)&BokNr (\textcolor{color1}{PK})\\
CHAR(5)&ISBN\\
CHAR(5)&Tittel\\
CHAR(5)&Forfatter\\
CHAR(5)&UtgittÅr\\
CHAR(5)&AntallKopier\\
CHAR(5)&Navn (\textcolor{color1}{FK})
}};

% Forlag
\pic[below=8em of A] {entity={C}{Forlag}{
CHAR(5)&Navn (\textcolor{color1}{PK})\\
CHAR(5)&Kontakt\\
CHAR(5)&Telefon
}};


% Utlån
\pic[right=10em of A] {entity={B}{Utlån}{
CHAR(5)&BokNr (\textcolor{color1}{PK, FK})\\
CHAR(5)&LNr (\textcolor{color1}{PK, FK})\\
CHAR(5)&UtlånsDato (\textcolor{color1}{PK})\\
CHAR(5)&ErLevert
}};

% Låner
\pic[below=8em of B] {entity={D}{Låner}{
CHAR(5)&LNr (\textcolor{color1}{PK})\\
CHAR(5)&Fornavn\\
CHAR(5)&Etternavn\\
CHAR(5)&Epost
}};

% Forhold
\draw[oone - omany] (A) edge  (B);
\draw[mone - omany, loosely dashed] (C) edge  (A);
\draw[mone - omany] (D) edge  (B);

% Forklaring
\pic {label={utgir}{A}{C}{right}};

\end{tikzpicture}
\end{center}

\begin{center}
\begin{tabular}{c}
\begin{lstlisting}[language=SQL]
-- SQL eksempelkode
SELECT * FROM ...;
CREATE TABLE Ordrelinje
(
    OrdreNr INTEGER,
    CNr CHAR(5),
    PrisPrEnhet DECIMAL(8,2),
    Antall INTEGER,
    CONSTRAINT OrdreLinjePN PRIMARY_KEY (OrdreNr, VNr)
    CONSTRAINT OrfreLinjeOrdreFN FOREIGN_KEY (OrdreNr) REFERENCES Ordre (OrdreNr)
    CONSTRAINT OrdrelinjeVareFN FOREIGN_KEY (VNr) REFERENCES Vare (VNr)
);
\end{lstlisting}
\end{tabular}
\end{center}

\section{Introduksjon}

Problemstilling: En forening har behov for et system for å administrere medlemmer. Det er tenkt å ta i bruk en SQL database for å bygge dette systemet.
Foreningen har flere lokallag, og alle medlemmer er med i nøyaktiv ett lokallag.
Lokallag har navn, leder (medlem), samt postnummer/sted og gatenavn/husnummer for møtelokale.
Et medlem har medlemsnummer, fornavn, etternavn, telefonnummer, e-postadresse, postnummer/sted, gatenavn/husnummer, og medlemsstatus (aktiv/utmeldt).
For hvert medlem skal det i tillegg loggføres om medlemsavgift hare blitt betalt for hvert år.\\

Oppgaver: Oppgaven vi skal ta for oss er å tegne en logisk (fullstendig) ER-modell i kråkefot-notasjon for en SQL database som fyller kravene spesifisert i problemstillingen, samt å vise at modellen tilfredsstiller 3. normalform.
Tegningen skal inneholde primærnøkler, fremmednøkler, datatyper, min/maks kardinalitet, sterke/svake entitetstyper og eksistensavhengighet/uavhengighet.

\section{Metode}

Vi starter med en felles diskusjon for hvordan vi ser for oss databasedesignet, for så å tegne en grov skisse av ER-modellen på whiteboard. Denne whiteboard-skissen tar vi bilde av for senere bruk.
Videre vil vi prøve å ta i bruk LaTeX-pakken TikZ for å utarbeide vår egen måte å tegne ER-diagram i kråkefot-notasjon på som tilfredsstiller alle oppgavekravene.
Vi ser for oss å bruke kode publisert på LaTeX Stack Exchange av AndréC\footnote{AndréC. (2018, 9. desember). \textit{How to create an ER diagram using tikzpicture environment}. LaTex Stack Exchange. \url{https://tex.stackexchange.com/questions/462914/how-to-create-an-er-diagram-using-tikzpicture-environment/463912#463912}} som utganspunkt for vår implementasjon.
Etter at vi har utarbeidet en egnet metode for ER-diagram tegning, blir neste steg å overføre whiteboard tegningen fra tidligere til LaTeX.
Når den endelige tegningen er ferdigstilt blir siste steg å skrive SQL spørringer for oppretting av databasen, samt å ta i bruk LaTeX-pakken listings for å legge ved SQL spørringer i den ferdige oppgaven. SQL spørringene tester vi underveis ved bruk av programmet DataGrip og en skybasert PostgreSQL database.



\vspace{0.5cm}
\begin{tabular}{|c|c|c|} 
\hline
&Fordel&Ulempe \\
\hline
    a&b&c\\
    a&b&c\\
    a&b&c\\
\hline
\end{tabular}

\section{Resultat}

TODO: Beskriv resultatet på deloppgavene etter å ha benyttet metodene


\section{Diskusjon}
TODO: Oppsummer resultatene, funnene og vurderingene


% - - - - - - - - - - - - - - - - - - - - - - - - - - - - - - - - - - - - - - - 

\end{document}
