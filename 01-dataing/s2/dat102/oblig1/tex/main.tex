\documentclass[norsk,8pt,a4paper]{report}
\usepackage[margin=1.5cm,tmargin=1.0cm,bmargin=1.0cm,rmargin=1.5cm,lmargin=1.5cm,footskip=0.2cm]{geometry}
\title{DAT102 - Obligatorisk innlevering 1}
\author{Gruppemedlemmer:\\Stephen Neba Fuh, Tord Johan Melheim,\\Ebubekir Siddik Yuksel, Casper Eide Özdemir-Børretzen }
\date{}

% % % % % % % % % % % % % % % % % % % % % % % % % % % % % % % % % % % % % % % % 

\usepackage{babel}               % Support for other languages than english
\usepackage{setspace}            % Set paragraph spacing
\usepackage{nicefrac}            % Nice looking fractals
\usepackage{graphicx}            % Images
\usepackage{gensymb}             % Degree symbol
\usepackage{listings}            % Code
\usepackage{ulem}                % Double underline
\usepackage{amssymb}             % ...
%\usepackage{pdfpages}            % Insert pdf pages
\usepackage{enumitem}            % Lists
\usepackage{colortbl}            % Colored tables
\usepackage[compact]{titlesec}   % ...
\usepackage[T1]{fontenc}         % ...
\usepackage[utf8]{inputenc}      % ...
\usepackage[fleqn]{amsmath}      % ...
\usepackage[makeroom]{cancel}    % ...
%\usepackage{empheq}              % ...
%\setstretch{1.5}
\setlength{\parindent}{0pt}
\titlespacing*{\subsection}{0cm}{1.5cm}{0.5cm}
\lstset{
aboveskip=0.1cm,
belowskip=1.0cm,
showstringspaces=false,
columns=flexible,
basicstyle={\scriptsize\ttfamily},
breaklines=true,
breakatwhitespace=true,
tabsize=4,
inputencoding = utf8,  % Input encoding
extendedchars = true,  % Extended ASCII
literate      =        % Support additional characters
{⋆}{{$\star\ $}}1 {∘}{{$\circ\ $}}1
{á}{{\'a}}1  {é}{{\'e}}1  {í}{{\'i}}1 {ó}{{\'o}}1  {ú}{{\'u}}1
{Á}{{\'A}}1  {É}{{\'E}}1  {Í}{{\'I}}1 {Ó}{{\'O}}1  {Ú}{{\'U}}1
{à}{{\`a}}1  {è}{{\`e}}1  {ì}{{\`i}}1 {ò}{{\`o}}1  {ù}{{\`u}}1
{À}{{\`A}}1  {È}{{\`E}}1  {Ì}{{\`I}}1 {Ò}{{\`O}}1  {Ù}{{\`U}}1
{ä}{{\"a}}1  {ë}{{\"e}}1  {ï}{{\"i}}1 {ö}{{\"o}}1  {ü}{{\"u}}1
{Ä}{{\"A}}1  {Ë}{{\"E}}1  {Ï}{{\"I}}1 {Ö}{{\"O}}1  {Ü}{{\"U}}1
{â}{{\^a}}1  {ê}{{\^e}}1  {î}{{\^i}}1 {ô}{{\^o}}1  {û}{{\^u}}1
{Â}{{\^A}}1  {Ê}{{\^E}}1  {Î}{{\^I}}1 {Ô}{{\^O}}1  {Û}{{\^U}}1
{œ}{{\oe}}1  {Œ}{{\OE}}1  {æ}{{\ae}}1 {Æ}{{\AE}}1  {ß}{{\ss}}1
{ẞ}{{\SS}}1  {ç}{{\c{c}}}1 {Ç}{{\c{C}}}1 {ø}{{\o}}1  {Ø}{{\O}}1
{å}{{\aa\ }}1  {Å}{{\AA}}1  {ã}{{\~a}}1  {õ}{{\~o}}1 {Ã}{{\~A}}1
{Õ}{{\~O}}1  {ñ}{{\~n}}1  {Ñ}{{\~N}}1  {¿}{{?`}}1  {¡}{{!`}}1
{„}{\quotedblbase}1 {“}{\textquotedblleft}1 {–}{$-$}1
{°}{{\textdegree}}1 {º}{{\textordmasculine}}1 {ª}{{\textordfeminine}}1
{£}{{\pounds}}1  {©}{{\copyright}}1  {®}{{\textregistered}}1
{«}{{\guillemotleft}}1  {»}{{\guillemotright}}1  {Ð}{{\DH}}1  {ð}{{\dh}}1
{Ý}{{\'Y}}1    {ý}{{\'y}}1    {Þ}{{\TH}}1    {þ}{{\th}}1    {Ă}{{\u{A}}}1
{ă}{{\u{a}}}1  {Ą}{{\k{A}}}1  {ą}{{\k{a}}}1  {Ć}{{\'C}}1    {ć}{{\'c}}1
{Č}{{\v{C}}}1  {č}{{\v{c}}}1  {Ď}{{\v{D}}}1  {ď}{{\v{d}}}1  {Đ}{{\DJ}}1
{đ}{{\dj}}1    {Ė}{{\.{E}}}1  {ė}{{\.{e}}}1  {Ę}{{\k{E}}}1  {ę}{{\k{e}}}1
{Ě}{{\v{E}}}1  {ě}{{\v{e}}}1  {Ğ}{{\u{G}}}1  {ğ}{{\u{g}}}1  {Ĩ}{{\~I}}1
{ĩ}{{\~\i}}1   {Į}{{\k{I}}}1  {į}{{\k{i}}}1  {İ}{{\.{I}}}1  {ı}{{\i}}1
{Ĺ}{{\'L}}1    {ĺ}{{\'l}}1    {Ľ}{{\v{L}}}1  {ľ}{{\v{l}}}1  {Ł}{{\L{}}}1
{ł}{{\l{}}}1   {Ń}{{\'N}}1    {ń}{{\'n}}1    {Ň}{{\v{N}}}1  {ň}{{\v{n}}}1
{Ő}{{\H{O}}}1  {ő}{{\H{o}}}1  {Ŕ}{{\'{R}}}1  {ŕ}{{\'{r}}}1  {Ř}{{\v{R}}}1
{ř}{{\v{r}}}1  {Ś}{{\'S}}1    {ś}{{\'s}}1    {Ş}{{\c{S}}}1  {ş}{{\c{s}}}1
{Š}{{\v{S}}}1  {š}{{\v{s}}}1  {Ť}{{\v{T}}}1  {ť}{{\v{t}}}1  {Ũ}{{\~U}}1
{ũ}{{\~u}}1    {Ū}{{\={U}}}1  {ū}{{\={u}}}1  {Ů}{{\r{U}}}1  {ů}{{\r{u}}}1
{Ű}{{\H{U}}}1  {ű}{{\H{u}}}1  {Ų}{{\k{U}}}1  {ų}{{\k{u}}}1  {Ź}{{\'Z}}1
{ź}{{\'z}}1    {Ż}{{\.Z}}1    {ż}{{\.z}}1    {Ž}{{\v{Z}}}1  {ž}{{\v{z}}}1
}

\newlength{\myeqskip}  \setlength{\myeqskip}{2pt}

% % % % % % % % % % % % % % % % % % % % % % % % % % % % % % % % % % % % % % % % 

\newcommand{\oppgave}[1]{\subsection*{Oppgave #1}}
\newcommand{\oppgaveDelStart}{\begin{enumerate}[leftmargin=*,itemsep=1.5cm,labelsep=1.5em,label=\alph*)]}
\newcommand{\oppgaveDelSlutt}{\end{enumerate}}
\newcommand{\oppgaveDel}[1]{\item[#1)]}

% % % % % % % % % % % % % % % % % % % % % % % % % % % % % % % % % % % % % % % % 

\begin{document}

% % % % % % % % % % % % % % % % % % % % % % % % % % % % % % % % % % % % % % % % 

\maketitle

\subsection*{Skjermbilder fra kjøring av Java program:}
\includegraphics[width=1.0\linewidth]{a4-screenshot_2026-02-01-105041.png}
\newpage
\includegraphics[width=1.0\linewidth]{a4-screenshot_2026-02-01-153637.png}
\newpage

% % % % % % % % % % % % % % % % % % % % % % % % % % % % % % % % % % % % % % % % 

\oppgave{3}
\oppgaveDelStart

\oppgaveDel{a}
\begin{alignat*}\\
i.   & \ \ 4n^2 + 50n - 10          &&\rightarrow \ O(n^2)\\
ii.  & \ \ 10n + 4 \log_2 n + 30    &&\rightarrow \ O(n)\\
iii. & \ \ 13n^3 + 22n^2 + 50n + 20 &&\rightarrow \ O(n^3)\\
iv.  & \ \ 35 + 13 \log_2 n         &&\rightarrow \ O(\log n)
\end{alignat*}

\oppgaveDel{b}
\begin{align*}
&\dots
\end{align*}

\oppgaveDel{c}
\begin{align*}
&\dots
\end{align*}

\oppgaveDel{d}
\begin{alignat*}
2 \pi r^2 \ &\rightarrow \ O(n^2)\\
2 \pi r   \ &\rightarrow \ O(n)
\end{alignat*}

\oppgaveDel{e}
Den ytre løkken har en tellevariabel "indeks" som går fra $0$ til $n-2$ og inkrementeres i hver iterasjon av løkken. Den ytre løkken kjøres derfor $n-1$ ganger.
Den indre løkken har en tellevariabel "igjen" som går fra indeks $+ \ 1$ til $n - 1$. Antall ganger den indre løkken kjøres er derfor avhengig av "indeks" verdien. Ved indeks $= 0$ må verdien som "indeks" refererer til sammenlignes med alle andre verdier i tabellen og den indre løkken kjøres $n - 1$ ganger. Ved indeks $= n-2$ har alle verdier i tabellen allerde blitt sammenlignet, bortsett fra den siste og nest-siste verdien, så den indre løkken kjøres bare $1$ gang for å sammenligne de to siste verdiene.\\

Det verste tilfelle for algoritmen er når tabellen ikke inneholder duplikater som fører til at alle verdier i tabellen må sammenlignes.\\

Eksempel ved $n=10$:
\begin{center}
\begin{tabular}{ l l l l }
indeks: & Ant. ganger indre løkke kjøres: & Totalt antall sammenligninger:\\ 
$0$ &  $(10-1) = 9$ & $9$\\
$1$ &  $(10-2) = 8$ & $17$\\
$2$ &  $(10-3) = 7$ & $24$\\
$3$ &  $(10-4) = 6$ & $30$\\
$4$ &  $(10-5) = 5$ & $35$\\
$5$ &  $(10-6) = 4$ & $39$\\
$6$ &  $(10-7) = 3$ & $42$\\
$7$ &  $(10-8) = 2$ & $44$\\
$8$ &  $(10-9) = 1$ & $45$\\
\end{tabular}
\end{center}

\newpage

Ved å lage et enkelt Java program kan antall sammenligner beregnes for andre $n$ verdier:\\

\begin{lstlisting}[language=Java]
public class Program {
    public static void main(String[] args) {
        System.out.println(" - - - - - - - - - - - - - - - - - - - - - - - - ");
        System.out.println(" OPPGAVE 3e ");
        System.out.println(" - - - - - - - - - - - - - - - - - - - - - - - - ");
        int[] nTab = new int[] { 10, 100, 1_000, 10_000, 100_000}; 
        for (int n : nTab) {
            print(n,oppg3e(n));
        }
    }

    private static void print(int n, int k) {
        System.out.printf("n = %d  -->  %d%n", n, k);
    }

    private static int oppg3e(int n) {
        int[] tab = new int[n];
        for (int i = 0; i < n; i++) {
            tab[i] = i;
        }
        // boolean harDuplikat( ... )
        // v = indeks for venstre tall i sammenligning
        // h = indeks for høyre tall i sammenligning
        // if (tab[v] == tab[h]) return true
        int c = 0; // c brukes for å telle antall sammenligninger
        for (int v = 0; v <= n - 2; v++) {
            for (int h = v + 1; h <= n - 1; h++) {
                c++; // c += 1
                if (tab[v] == tab[h]) {
                    return c; // return true her (to like tall funnet)
                }
            }
        }
        return c; // return false her (ingen like tall funnet)
    }

}
\end{lstlisting}

Dette gir følgende resultat:\\

\begin{lstlisting}
 - - - - - - - - - - - - - - - - - - - - - - - -
 OPPGAVE 3e
 - - - - - - - - - - - - - - - - - - - - - - - -
n = 10  -->  45
n = 100  -->  4950
n = 1000  -->  499500
n = 10000  -->  49995000
n = 100000  -->  704982704
\end{lstlisting}

Antall sammenligninger i forhold til $n$ kan beskrives som en aritmetisk rekke $s_m = \frac{m \cdot (a_1 + a_m)}{2}$.
Der $m$ er antall ganger den ytre løkken kjøres $(n-1)$. $a_1$ er første tall i rekken $(n-1)$. Og $a_n$ er siste tall i rekken $(1)$. Det gir s_{n-1} = \frac{(n-1) \cdot ((n-1) + 1)}{2} = \frac{n^2 - n}{2}$, som i O-notasjon er $O(n^2)$.

\begin{alignat*}\\
&\text{}
\end{alignat*}

\oppgaveDel{f}
\begin{alignat*}{2}
i.   & \ \ t_1(n) = 8n + 4n^3              &&\rightarrow \ O(n^3)\\
ii.  & \ \ t_2(n) = 10 \log_2 n + 20       &&\rightarrow \ O(\log n)\\
iii. & \ \ t_3(n) = 20n + 2n \log_2 n + 11 &&\rightarrow \ O(n \log n)\\
iv.  & \ \ t_4(n) = 4 \log_2 n + 2n        &&\rightarrow \ O(2^n)\\\\
&\text{Rangert etter effektivitet:}\\
1) & \ \ O(\log n)  &&[t_2]\\
2) & \ \ O(n \log n)&&[t_3]\\
3) & \ \ O(n^3)     &&[t_1]\\
4) & \ \ O(2^n)     &&[t_4]
\end{alignat*}

\oppgaveDel{g}

Hvis vi bruker antall tilordninger som måleenhet for algoritmen slik som tidligere, ser vi at det alltid gjennomføres to tilordninger for \lstinline{long k = 0} og \lstinline{long i = 1}. For-løkken kjøres $n-1$ ganger, og for hver iterasjon gjennomføres to tilordninger, \lstinline{i++} og \lstinline{k = k + 5}.\\

Antall tilordninger, for $n > 0 \rightarrow 2 + 2(n-1) = 2n$. Som er $O(n)$ i O-notasjon.\\

Ved å lage et enkelt Java-program kan vi måle tiden for $n=10^7, 10^8$ og $10^9$:\\

\begin{lstlisting}[language=Java]
public class Program {
    public static void main(String[] args) {
        System.out.println(" - - - - - - - - - - - - - - - - - - - - - - - - ");
        System.out.println(" OPPGAVE 3g ");
        System.out.println(" - - - - - - - - - - - - - - - - - - - - - - - - ");
        long[] nTab = new long[] { 10_000_000, 100_000_000, 1_000_000_000};
        for (long n : nTab) {
            int ant = 20;
            long res = 0L;
            for (int i = 0; i < ant; i++) {
                long start = System.currentTimeMillis();
                long k = 0;
                for (long j = 1; j <= n; j++) {
                    k = k + 5;
                }
                res += System.currentTimeMillis() - start;
            }
            System.out.printf("n = %d  -->  %.1f ms%n", n, (double)res / (double)ant);
        }
    }
}
\end{lstlisting}

Dette gir følgende resultat:\\
\begin{lstlisting}
 - - - - - - - - - - - - - - - - - - - - - - - -
 OPPGAVE 3g
 - - - - - - - - - - - - - - - - - - - - - - - -
n = 10000000  -->  5.0 ms
n = 100000000  -->  47.0 ms
n = 1000000000  -->  465.7 ms
\end{lstlisting}

Resultatene viser en lineær økning i tiden det tar å gjennomføre metoden, som stemmer overens med hva vi hadde forventet ved $O(n)$.\\

Vekstfunksjonen i dette tilfelle er $T(n) = cn$, der $c \approx 0.0000005 \ ms$. $c$ vil variere avhengig av faktorer som hvilken maskinvare og operativsystem programmet kjører på. I tillegg til en rekke andre faktorer slik som om det er andre prossesser som kjører samtidig, hvilken versjon av Java som brukes og om det skal gjennomføres søppeltømming underveis.

\oppgaveDelSlutt

% % % % % % % % % % % % % % % % % % % % % % % % % % % % % % % % % % % % % % % % 

\end{document}

