\documentclass[norsk,8pt,a4paper]{report}
\usepackage[margin=1.5cm,tmargin=1.0cm,bmargin=1.0cm,rmargin=1.5cm,lmargin=1.5cm,footskip=0.2cm]{geometry}
\title{MAT101 - Innlevering 1 (Gruppe 16)}
\author{}
\date{}

% % % % % % % % % % % % % % % % % % % % % % % % % % % % % % % % % % % % % % % % 

\usepackage{babel}               % Support for other languages than english
\usepackage{setspace}            % Set paragraph spacing
\usepackage{nicefrac}            % Nice looking fractals
\usepackage{graphicx}            % Images
\usepackage{gensymb}             % Degree symbol
\usepackage{listings}            % Code
\usepackage{ulem}                % Double underline
\usepackage{amssymb}             % ...
\usepackage{pdfpages}            % Insert pdf pages
\usepackage{enumitem}            % Lists
\usepackage{colortbl}            % Colored tables
\usepackage[compact]{titlesec}   % ...
\usepackage[T1]{fontenc}         % ...
\usepackage[utf8]{inputenc}      % ...
\usepackage[fleqn]{amsmath}      % ...
\usepackage[makeroom]{cancel}    % ...
%\usepackage{empheq}              % ...
%\setstretch{1.5}
\setlength{\parindent}{0pt}
\titlespacing*{\subsection}{0cm}{1.5cm}{0.5cm}
\lstset{
aboveskip=0cm,
belowskip=0cm,
showstringspaces=false,
columns=flexible,
basicstyle={\scriptsize\ttfamily},
breaklines=true,
breakatwhitespace=true,
tabsize=4,
inputencoding = utf8,  % Input encoding
extendedchars = true,  % Extended ASCII
literate      =        % Support additional characters
{⋆}{{$\star\ $}}1 {∘}{{$\circ\ $}}1
{á}{{\'a}}1  {é}{{\'e}}1  {í}{{\'i}}1 {ó}{{\'o}}1  {ú}{{\'u}}1
{Á}{{\'A}}1  {É}{{\'E}}1  {Í}{{\'I}}1 {Ó}{{\'O}}1  {Ú}{{\'U}}1
{à}{{\`a}}1  {è}{{\`e}}1  {ì}{{\`i}}1 {ò}{{\`o}}1  {ù}{{\`u}}1
{À}{{\`A}}1  {È}{{\`E}}1  {Ì}{{\`I}}1 {Ò}{{\`O}}1  {Ù}{{\`U}}1
{ä}{{\"a}}1  {ë}{{\"e}}1  {ï}{{\"i}}1 {ö}{{\"o}}1  {ü}{{\"u}}1
{Ä}{{\"A}}1  {Ë}{{\"E}}1  {Ï}{{\"I}}1 {Ö}{{\"O}}1  {Ü}{{\"U}}1
{â}{{\^a}}1  {ê}{{\^e}}1  {î}{{\^i}}1 {ô}{{\^o}}1  {û}{{\^u}}1
{Â}{{\^A}}1  {Ê}{{\^E}}1  {Î}{{\^I}}1 {Ô}{{\^O}}1  {Û}{{\^U}}1
{œ}{{\oe}}1  {Œ}{{\OE}}1  {æ}{{\ae}}1 {Æ}{{\AE}}1  {ß}{{\ss}}1
{ẞ}{{\SS}}1  {ç}{{\c{c}}}1 {Ç}{{\c{C}}}1 {ø}{{\o}}1  {Ø}{{\O}}1
{å}{{\aa\ }}1  {Å}{{\AA}}1  {ã}{{\~a}}1  {õ}{{\~o}}1 {Ã}{{\~A}}1
{Õ}{{\~O}}1  {ñ}{{\~n}}1  {Ñ}{{\~N}}1  {¿}{{?`}}1  {¡}{{!`}}1
{„}{\quotedblbase}1 {“}{\textquotedblleft}1 {–}{$-$}1
{°}{{\textdegree}}1 {º}{{\textordmasculine}}1 {ª}{{\textordfeminine}}1
{£}{{\pounds}}1  {©}{{\copyright}}1  {®}{{\textregistered}}1
{«}{{\guillemotleft}}1  {»}{{\guillemotright}}1  {Ð}{{\DH}}1  {ð}{{\dh}}1
{Ý}{{\'Y}}1    {ý}{{\'y}}1    {Þ}{{\TH}}1    {þ}{{\th}}1    {Ă}{{\u{A}}}1
{ă}{{\u{a}}}1  {Ą}{{\k{A}}}1  {ą}{{\k{a}}}1  {Ć}{{\'C}}1    {ć}{{\'c}}1
{Č}{{\v{C}}}1  {č}{{\v{c}}}1  {Ď}{{\v{D}}}1  {ď}{{\v{d}}}1  {Đ}{{\DJ}}1
{đ}{{\dj}}1    {Ė}{{\.{E}}}1  {ė}{{\.{e}}}1  {Ę}{{\k{E}}}1  {ę}{{\k{e}}}1
{Ě}{{\v{E}}}1  {ě}{{\v{e}}}1  {Ğ}{{\u{G}}}1  {ğ}{{\u{g}}}1  {Ĩ}{{\~I}}1
{ĩ}{{\~\i}}1   {Į}{{\k{I}}}1  {į}{{\k{i}}}1  {İ}{{\.{I}}}1  {ı}{{\i}}1
{Ĺ}{{\'L}}1    {ĺ}{{\'l}}1    {Ľ}{{\v{L}}}1  {ľ}{{\v{l}}}1  {Ł}{{\L{}}}1
{ł}{{\l{}}}1   {Ń}{{\'N}}1    {ń}{{\'n}}1    {Ň}{{\v{N}}}1  {ň}{{\v{n}}}1
{Ő}{{\H{O}}}1  {ő}{{\H{o}}}1  {Ŕ}{{\'{R}}}1  {ŕ}{{\'{r}}}1  {Ř}{{\v{R}}}1
{ř}{{\v{r}}}1  {Ś}{{\'S}}1    {ś}{{\'s}}1    {Ş}{{\c{S}}}1  {ş}{{\c{s}}}1
{Š}{{\v{S}}}1  {š}{{\v{s}}}1  {Ť}{{\v{T}}}1  {ť}{{\v{t}}}1  {Ũ}{{\~U}}1
{ũ}{{\~u}}1    {Ū}{{\={U}}}1  {ū}{{\={u}}}1  {Ů}{{\r{U}}}1  {ů}{{\r{u}}}1
{Ű}{{\H{U}}}1  {ű}{{\H{u}}}1  {Ų}{{\k{U}}}1  {ų}{{\k{u}}}1  {Ź}{{\'Z}}1
{ź}{{\'z}}1    {Ż}{{\.Z}}1    {ż}{{\.z}}1    {Ž}{{\v{Z}}}1  {ž}{{\v{z}}}1
}

\newlength{\myeqskip}  \setlength{\myeqskip}{2pt}
\AtBeginDocument{%
    \setlength\abovedisplayskip{\myeqskip}%
    \setlength\belowdisplayskip{\myeqskip}%
    \setlength\abovedisplayshortskip{\myeqskip-\baselineskip}%
    \setlength\belowdisplayshortskip{\myeqskip}}

% % % % % % % % % % % % % % % % % % % % % % % % % % % % % % % % % % % % % % % % 

\newcommand{\oppgave}[1]{\subsection*{Oppgave #1}}
\newcommand{\oppgaveDelStart}{\begin{enumerate}[leftmargin=*,itemsep=1.5cm,labelsep=1.5em,label=\alph*)]}
\newcommand{\oppgaveDelSlutt}{\end{enumerate}}
\newcommand{\oppgaveDel}[1]{\item[#1)]}

% % % % % % % % % % % % % % % % % % % % % % % % % % % % % % % % % % % % % % % % 

\begin{document}

% % % % % % % % % % % % % % % % % % % % % % % % % % % % % % % % % % % % % % % % 

\oppgave{1}
\begin{align*}{}
&A = \{x \in \mathbb{Z} | (2x - 1)(2x + 1) = 0\}\\
&B = \{x \in \mathbb{R} | (2x - 1)(2x + 1) = 0\}\\\\
&\text{Gitt at\ } (2x-1)(2x+1) = 0\\
&\text{da må\ } 2x - 1 = 0 \Rightarrow x = \nicefrac{1}{2}\\
&\text{eller\ } 2x + 1 = 0 \Rightarrow x = -\nicefrac{1}{2}\\
\end{align*}
\oppgaveDelStart

\oppgaveDel{a}
\begin{align*}
&A = \emptyset\\
&\text{Siden $\nicefrac{1}{2}$\ og\ $-\nicefrac{1}{2}$ ikke er elementer i mengden $\mathbb{Z}$.}
\end{align*}

\oppgaveDel{b}
\begin{align*}
&B = \{-\nicefrac{1}{2},\nicefrac{1}{2}\}
\end{align*}

\oppgaveDelSlutt

% % % % % % % % % % % % % % % % % % % % % % % % % % % % % % % % % % % % % % % % 

\oppgave{2}
\begin{align*}
&p \star q = \sim p \land \sim q\\
&p \circ q = \sim p \lor \sim q\\
\end{align*}
\oppgaveDelStart

\oppgaveDel{a}
\begin{tabular}{ |c|c|c|c|c|c| } 
 \hline
 $p$ & $q$ & $\sim p$ & $\sim q$ & $p \star q$ & $p \circ q$ \\ 
 \hline
 T & T & F & F & F & F  \\ 
 T & F & F & T & F & T  \\ 
 F & T & T & F & F & T  \\ 
 F & F & T & T & T & T  \\ 
 \hline
\end{tabular}

\oppgaveDel{b}
\begin{lstlisting}
-- Nøstet liste der det første elementet representerer 
-- sannhetsvariabelen p og det andre elementet representerer q
pq_tabell = [[True,True],[True,False],[False,True],[False,False]]

-- Hent første og andre element fra den gitte listen,
-- gjennomfør negasjon på elementene og returner boolsk verdi 
-- etter gjennomført "AND" binær operasjon
star :: [Bool] -> Bool
star pq = (not.head)(pq) && (not.last)(pq) 

-- Hent første og andre element fra den gitte listen,
-- gjennomfør negasjon på elementene og returner boolsk verdi 
-- etter gjennomført "OR" binær operasjon
circ :: [Bool] -> Bool
circ pq = (not.head)(pq) || (not.last)(pq) 

-- main funksjon for å kjøre koden
main :: IO()
main = do
    putStrLn "p ⋆ q: "
    -- Skriv ut star funksjonen med pq_tabellen som parameter
    print(map star pq_tabell)
    putStrLn "p ∘ q: "
    -- Skriv ut circ funksjonen med pq_tabellen som parameter
    print(map circ pq_tabell)

-- Utputt ved kjørt program:
-- p ⋆ q:
-- [False,False,False,True]
-- p ∘ q:
-- [False,True,True,True]
\end{lstlisting}

\newpage

\oppgaveDel{c}
\begin{align*}
&\sim p\ \circ \sim q = \sim ( \sim p ) \lor \sim ( \sim q) = p \lor q
\end{align*}
\begin{tabular}{ |c|c|c|c|c|>{\columncolor[gray]{0.9}}c|>{\columncolor[gray]{0.9}}c| } 
 \hline
 $p$ & $q$ & $\sim p$ & $\sim q$ & $p \star q$ & $\sim (p \star q)$ & $\sim p\ \circ \sim q$ \\ 
 \hline
 T & T & F & F & F & T & T  \\ 
 T & F & F & T & F & T & T \\ 
 F & T & T & F & F & T & T \\ 
 F & F & T & T & T & F & F \\ 
 \hline
\end{tabular}
\begin{align*}
&\text{De to kolonnene har samme verdier og logisk ekvivalens er bevist.}\\
&\text{Dette kan uttrykkes symbolsk ved\ } \sim (p \star q) \equiv \sim p\ \circ \sim q\\
\end{align*}

\oppgaveDel{d}
\begin{align*}
\sim (p \star q) &\equiv p \lor q\\
\sim (\sim p \land \sim q) &\equiv \text{Fra definisjonen av\ }p \star q\\
\sim (\sim p) \lor \sim (\sim q) &\equiv \text{De Morgans lov}\\
p \lor q &\equiv \text{Dobbel negasjons lov}\\
\end{align*}

\begin{align*}
\sim p\ \circ \sim q &\equiv p \lor q\\
\sim (\sim p) \land \sim (\sim q) &\equiv \text{Fra definisjonen av\ }p \circ q\\
p \lor q &\equiv \text{Dobbel negasjons lov}\\
\end{align*}

\oppgaveDelSlutt

% % % % % % % % % % % % % % % % % % % % % % % % % % % % % % % % % % % % % % % % 

\oppgave{3}
\oppgaveDelStart

\oppgaveDel{a}
\begin{tabular}{ |c|c|c|>{\columncolor[gray]{0.9}}c|c|c|>{\columncolor[gray]{0.9}}c| } 
 \hline
 $p$ & $q$ & $\sim p$ & $\sim q$ & $(p \lor \sim q)$ & $(q \rightarrow \sim p)$ & $(p \lor \sim q) \land (q \rightarrow \sim p)$ \\
 \hline
 T & T & F & F & T & F & F\\
 T & F & F & T & T & T & T\\
 F & T & T & F & F & T & F\\
 F & F & T & T & T & T & T\\
 \hline
\end{tabular}

\oppgaveDel{b}
\begin{align*}
(p \lor \sim q) \land (q \rightarrow \sim p) &\equiv \sim q\\
(p \lor \sim q) \land (\sim q \lor \sim p) &\equiv \text{Definisjonen av implikasjon}\\
\sim q \lor (p \land \sim p) &\equiv \text{Distributiv lov}\\
\text{Utsagnet $(p \land \sim p)$}&\text{\ er en selvmotsigelse}\\
\sim q &\equiv \text{Identitetslov}\\
\end{align*}

\oppgaveDelSlutt

% % % % % % % % % % % % % % % % % % % % % % % % % % % % % % % % % % % % % % % % 

\newpage
\oppgave{4}
\begin{align*}
&A = \{1,2,...,10\}\\
&R = \{(x,y)\ |\ x + y \le 3\}, x \in A, y \in A
\end{align*}
\oppgaveDelStart

\oppgaveDel{a}
\begin{align*}
&X = \{1,2\}\\
&Y = \{1,2\}
\end{align*}

\oppgaveDel{b}
\begin{align*}
&X \times Y = \{(1,1),(1,2),(2,1),(2,2)\}
\end{align*}

\oppgaveDel{c}
\begin{align*}
&R = \{(1,1),(1,2),(2,2)\}
\end{align*}

\oppgaveDel{d}
\begin{lstlisting}
-- Importer Data.List for å kunne bruke nub funksjonen
import Data.List

-- Definer mengden A
a = [1..10]

-- Definer mengden R
r = [[x,y]| x <- a, y <- a, x + y <= 3]

-- Definer mengden X som første elementet fra 
-- den nøstede listen i mengden R og fjern duplikater (nub)
x = (nub)(map head r)

-- Et alternativ til nub, for å unngå duplikater i mengden X
-- kunne vært å ta i bruk Haskells innebygde Set datastruktur ved:
-- import qualified Data.Set as Set
-- x = (Set.fromList)(map head r)

-- main funksjon for å kjøre koden
main :: IO()
main = do
    putStrLn "Mengden X: "
    print(x)

-- Utputt ved kjørt program:
-- Mengden X:
-- [1,2]
\end{lstlisting}

\oppgaveDel{e}
\begin{align*}
&B = \{1,2,...,20\}\\
&S = \{(x,y)\ |\ 2x + y = 7, x \in B, y \in B\}\\
&S = \{(1,5),(2,3),(3,1)\}
\end{align*}

\oppgaveDel{f}
\begin{lstlisting}
-- Definer mengden B
b = [1..20]

-- Definer mengden S
s = [[x,y]| x <- b, y <- b, 2 * x + y == 7]

-- main funksjon for å kjøre koden
main :: IO()
main = do
    putStrLn "Mengden S: "
    print(s)

-- Utputt ved kjørt program:
-- Mengden S:
-- [[1,5],[2,3],[3,1]]
\end{lstlisting}

\oppgaveDelSlutt

% % % % % % % % % % % % % % % % % % % % % % % % % % % % % % % % % % % % % % % % 

\newpage
\oppgave{5}
\begin{lstlisting}

-- t1 er de naturlige tallene fra 1 til 10
t1 = [1..10]

-- t2 er de naturlige tallene fra 11 til 20
t2 = [11..20]

-- tupler er mengden av tupler med det første elementet fra t1 og det andre fra t2
-- det første elementet x kommer fra iterasjon over "mengden" (listen) t1
-- det andre elementet hentes fra t2 med (x-1) som indeks
tupler = [(x,t2!!(x-1)) | x <- t1]

-- main funksjon for å kjøre koden
main :: IO()
main = do
    putStrLn "Resultat: "
    print(tupler)

-- Utputt ved kjørt program:
-- Resultat:
-- [(1,11),(2,12),(3,13),(4,14),(5,15),(6,16),(7,17),(8,18),(9,19),(10,20)]
\end{lstlisting}

% % % % % % % % % % % % % % % % % % % % % % % % % % % % % % % % % % % % % % % % 

\oppgave{6}
\oppgaveDelStart

\oppgaveDel{a}
Det eksisterer noe som, hvis det er god kaffi, da er det cappuccino.

\oppgaveDel{b}
Hvis noe ikke er godt, da er det kaffi og det er cappuccino.

\oppgaveDel{c}
\begin{align*}
&\forall x: \Big( R(x) \rightarrow P(x) \land Q(x) \Big)
\end{align*}

\oppgaveDelSlutt

% % % % % % % % % % % % % % % % % % % % % % % % % % % % % % % % % % % % % % % % 

\newpage
\oppgave{7}
\oppgaveDelStart

\oppgaveDel{a}
\begin{align*}
&\forall x \in \mathbb{Z},\ \exists k \in \mathbb{Z},\ x \cdot k < 0
\end{align*}

\oppgaveDel{b}
\begin{align*}
&\text{La $E(x)$ bety ''$x$ er et partall''}\\
&\forall x \in \mathbb{Z}, \forall y \in \mathbb{Z}, \Big( E(x) \rightarrow E(x \cdot y)\Big)
\end{align*}

\begin{align*}
&\text{Alternativ løsning:}\\
&\forall x \in \mathbb{Z}, \forall y \in \mathbb{Z}, \Big( \exists m \in \mathbb{Z}, x = 2m\Big) \rightarrow \Big( \exists n \in \mathbb{Z}, x \cdot y = 2n \Big)
\end{align*}


\oppgaveDel{c}
\begin{align*}
&\exists x \in \mathbb{Z}, \sqrt{x} > 2
\end{align*}

\oppgaveDel{d}
\begin{align*}
&\exists n \in \mathbb{R}, \exists m \in \mathbb{R}, n^2 = 4 \land m^2 = 4\\
\end{align*}

\oppgaveDelSlutt

% % % % % % % % % % % % % % % % % % % % % % % % % % % % % % % % % % % % % % % % 

\end{document}

